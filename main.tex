\documentclass[a4paper,12pt]{article}

\usepackage{hyperref}
\usepackage[affil-it]{authblk}
\usepackage[rgb]{xcolor}
\hypersetup{				% Гиперссылки
    colorlinks=true,       	% false: ссылки в рамках
	urlcolor=blue           % на URL
}

%  Русский язык
\usepackage[T2A]{fontenc}			% кодировка
			% кодировка исходного текста
\usepackage[english,russian]{babel}	% локализация и переносы
\usepackage[utf8]{inputenc}

% Математика
\usepackage{amsmath,amsfonts,amssymb,amsthm,mathtools}


\usepackage{wasysym}

%Заголовок
\author{Мирзоев С.Е.%
\thanks{E-mail: \texttt{mrsergeymirzoev@gmail.com}}}
\title{Оценка бессрочных опционов для случайных процессов со скачками}
\affil{Факультет Вычислительной Математики и Кибернетики, Московский государственный университет им. М.В.Ломоносова}

\makeatother

\begin{document}

\begin{titlepage}
\centering

\textsc{Дипломная работа}

\vspace{\stretch{1}}

{\LARGE\bfseries Оценка бессрочных опционов для случайных процессов со скачками\\}
\rule{3in}{0.4pt}

\vspace{\stretch{1}}

Мирзоев С.Е.\\
Московский государственный университет им. М.В.Ломоносова\\
Факультет Вычислительной Математики и Кибернетики\\
Октябрь 2021

\vspace{\stretch{1}}

{\small
Научный руководитель: Белянкин Г.А.\\
Со-руководитель: Морозов В.В.}

\vspace*{\stretch{1}}

\end{titlepage}
\thispagestyle{empty}

\newpage
\tableofcontents

\newpage

\section{Вступление}

В статье Гербера и Шиу была сделана оценка бесконечного пут-опциона, в данной статье аналогичный результат будет повторён для колл-опциона. В указанной статье рассматриваются две модели, в которых логарифм цены актива представляет собой сдвинутый составной пуассоновский процесс. Явные результаты получены для цен и оптимальных стратегий исполнения определенных бессрочных американских опционов на актив, в частности для бессрочного пут-опциона. В первой модели, в которой скачки цены актива направлены вверх, результаты получаются с использованием процесса - мартингала и (условия плавного перехода?). Во второй модели, в которой скачки идут вниз, показывалось, что значение стратегии, соответствующее постоянной границе исполнения опциона удовлетворяет определенному уравнению обновления. Затем из (условия непрерывного перехода?) получается оптимальная стратегия исполнений. Кроме того, одна и та же модель может быть использована для определения цены на определенные параметры отмены. Наконец, было показано, как классическая модель геометрического броуновского движения может быть использована в качестве предела, а также, как она может быть встроена в обе модели.

\section{Введение}


\section{Сопутствующая литература}

\begin{thebibliography}{3}
\bibitem{GerberShiu}
Hans U. Gerber and Elias S.W. Shiu \textbf{"Pricing Perpetual Options for Jump Processes"} // --- North American Actuarial Journal, 1998 --- vol. 2, issue 3, 108-109 ---
\end{thebibliography}

\section{Модель}


\section{Бесконечный колл-опцион}


\section{Заключение}


\end{document}
