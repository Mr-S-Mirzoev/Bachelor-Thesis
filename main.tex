\documentclass[a4paper,12pt]{article}

\usepackage{hyperref}
\usepackage[affil-it]{authblk}
\usepackage[rgb]{xcolor}
\hypersetup{				% Гиперссылки
    colorlinks=true,       	% false: ссылки в рамках
	urlcolor=blue           % на URL
}

%  Русский язык
\usepackage[T2A]{fontenc}			% кодировка
			% кодировка исходного текста
\usepackage[english,russian]{babel}	% локализация и переносы
\usepackage[utf8]{inputenc}

% Математика
\usepackage{amsmath,amsfonts,amssymb,amsthm,mathtools}

\usepackage[colorlinks]{hyperref}
\usepackage{cleveref}

% Mathematical formating
\newtheorem{theorem}{Теорема}
\theoremstyle{definition}
\newtheorem{definition}{Определение}[section]
\renewcommand{\theoremautorefname}{теоремы}

\newenvironment{delayedproof}[1]
 {\begin{proof}[\raisedtarget{#1}Доказательство \autoref{#1}]}
 {\end{proof}}
\newcommand{\raisedtarget}[1]{%
  \raisebox{\fontcharht\font`P}[0pt][0pt]{\hypertarget{#1}{}}%
}
\newcommand{\proofref}[1]{\hyperlink{#1}{в приложении к статье}}

\usepackage{wasysym}

%Заголовок
\author{Мирзоев С.Е.%
\thanks{E-mail: \texttt{mrsergeymirzoev@gmail.com}}}
\title{Оценка бессрочных опционов для случайных процессов со скачками}
\affil{Факультет Вычислительной Математики и Кибернетики, Московский государственный университет им. М.В.Ломоносова}

\makeatletter
% Roman numbers support
\newcommand*{\rom}[1]{\expandafter\@slowromancap\romannumeral #1@}

\renewcommand\qedsymbol{$\blacksquare$}

\makeatother

\begin{document}

\begin{titlepage}
\centering

\textsc{Дипломная работа}

\vspace{\stretch{1}}

{\LARGE\bfseries Оценка бессрочных опционов для случайных процессов со скачками\\}
\rule{3in}{0.4pt}

\vspace{\stretch{1}}

Мирзоев С.Е.\\
Московский государственный университет им. М.В.Ломоносова\\
Факультет Вычислительной Математики и Кибернетики\\
Октябрь 2021

\vspace{\stretch{1}}

{\small
Научный руководитель: Белянкин Г.А.\\
Со-руководитель: Морозов В.В.}

\vspace*{\stretch{1}}

\end{titlepage}
\thispagestyle{empty}

\newpage
\tableofcontents

\newpage

\section{Вступление}

В статье Гербера и Шиу была сделана оценка бесконечного пут-опциона, в данной статье аналогичный результат будет повторён для колл-опциона. В указанной статье рассматриваются две модели, в которых логарифм цены актива представляет собой сдвинутый составной пуассоновский процесс. Явные результаты получены для цен и оптимальных стратегий исполнения определенных бессрочных американских опционов на актив, в частности для бессрочного пут-опциона. В первой модели, в которой скачки цены актива направлены вверх, результаты получаются с использованием процесса - мартингала и (условия плавного перехода?). Во второй модели, в которой скачки идут вниз, показывалось, что значение стратегии, соответствующее постоянной границе исполнения опциона удовлетворяет определенному уравнению обновления. Затем из (условия непрерывного перехода?) получается оптимальная стратегия исполнения. Кроме того, одна и та же модель может быть использована для определения цены на определенные параметры отмены. Наконец, было показано, как классическая модель геометрического броуновского движения может быть использована в качестве предела, а также, как она может быть встроена в обе модели.

\section{Введение}

Пусть S(t) - цена актива, например, акции, в момент времени $t \ge 0$. Мы предполагаем, что рынок риск-нейтрален, следовательно, стоимость ценной бумаги - это математическое ожидание дисконтированных будущих платежей. В предположении, что актив не выплачивает никаких дивидендов и что безрисковая мгновенная процентная ставка $r$ является положительной константой, случайный процесс дисконтированной цены актива $\{e^{-rt} S(t)\}$ является мартингалом.

Пусть $U(t) = ln S(t), t = 0$. Мы предполагаем, что $\{U(t), t \ge 0\}$ - процесс с независимыми и стационарными приращениями, с начальным значением $U(0) = u$. Классическая модель цены акций является частным случаем, в котором $\{U(t)\}$ представляет собой винеровский процесс со сдвигом $\mu = r - \frac{\sigma ^ {2}}{2}$ и бесконечно малой дисперсией $\sigma ^ {2}$. 

В данной работе будут рассмотрены две модели:

Модель \rom{1}: $U(t) = u - ct + Z(t)$

Модель \rom{2}: $U(t) = u + ct - Z(t)$

В обоих случаях $c > 0$ является постоянной величиной, а $\{Z(t)\}$ представляет собой составной пуассоновский процесс, определяемый параметром $\lambda > 0$ и распределением величин скачка. Для упрощения обозначений предположим, что распределения величин скачка
непрерывны с плотностью вероятности $p(x), x \ge 0$. 

\begin{theorem}[Отсутствие арбитража]\label{thm:no_arbitrage_theoreme}
В модели \rom{2} мы делаем дополнительное предположение, что $c > r$ иначе возникает возможность арбитража. 
\end{theorem}
\begin{proof}
\proofref{thm:no_arbitrage_theoreme}
\end{proof}

Обратите внимание, что модель \rom{2} напоминает модель, которая используется для процесса избытка в классической теории риска.

\section{Сопутствующая литература}

\begin{thebibliography}{3}
\bibitem{GerberShiu}
Hans U. Gerber and Elias S.W. Shiu \textbf{"Pricing Perpetual Options for Jump Processes"} // --- North American Actuarial Journal, 1998 --- vol. 2, issue 3, 108-109 ---
\end{thebibliography}

\section{Модель}


\section{Бесконечный колл-опцион}


\section{Заключение}

\appendix
\section{Выкладки}
\begin{delayedproof}{thm:no_arbitrage_theoreme} 
Если бы было допустимо $c \le r$, мы бы имели $S(t) \le S(0)e^{rt}$ со строгим неравенством после первого скачка. Следовательно, путем короткой продажи (Подробнее: \autoref{def:short_position}) актива и инвестирования выручки S(0) по безрисковой процентной ставке r мы могли бы получить безрисковую прибыль:

В момент 0 инвестор берёт актив в короткую позицию, сразу её продаёт и сумму $S(0)$ кладёт на депозит в банк под ставку $r$. Далее он ждёт, когда выполнится $e^{rt} S(0) > S(t)$ (это произойдёт при первом скачке стоимости актива $S(t)$. Инвестор забирает деньги с банковского депозита, покупает актив, возвращает его брокеру и имеет безрисковую прибыль $e^{rt}S(0) - S(t)$
\end{delayedproof}

\section{Определения}
\begin{definition}[Шорт, короткая позиция]
    \label{def:short_position}
    \textbf{Продажа без покрытия или торговля в шорт} - продажа ценных бумаг, товаров или валюты, которыми торговец на момент продажи не владеет. Такая операция возможна, если условия контракта предусматривают его исполнение (поставку) через некоторое время или при маржинальной торговле, когда разрешено продавать взятый у брокера в кредит товар с предполагаемой последующей покупкой аналогичного товара и возврата кредита в натуральном (товарном) виде. Торговец надеется, что цена упадёт и он сможет дешевле выкупить ранее проданный товар. Этот механизм обеспечивает возможность получать прибыль при снижении цен.
\end{definition}

\end{document}
