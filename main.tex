 \documentclass[a4paper,12pt]{article}

\usepackage[colorlinks,unicode]{hyperref}
\usepackage[affil-it]{authblk}
\usepackage[rgb]{xcolor}
\hypersetup{				% Гиперссылки
    colorlinks=true,       	% false: ссылки в рамках
	urlcolor=blue           % на URL
}

%  Русский язык
\usepackage[T2A]{fontenc}			% кодировка
			% кодировка исходного текста
\usepackage[english,russian]{babel}	% локализация и переносы
\usepackage[utf8]{inputenc}

% Footnotes
\usepackage[bottom]{footmisc}

% Математика
\usepackage{amsmath,amsfonts,amssymb,amsthm,mathtools}
\usepackage{thmtools}

\usepackage{physics}
\usepackage{cleveref}

% Mathematical formating
\newtheorem{theorem}{Теорема}
\theoremstyle{definition}
\newtheorem{definition}{Определение}[section]
\renewcommand{\theoremautorefname}{теоремы}

% Add lemma support
\newtheorem{lemma}{Лемма}
\newtheorem{sublemma}{Лемма}[lemma]

\usepackage{wasysym}

%Заголовок
\author{Мирзоев С.Е.%
\thanks{E-mail: \texttt{mrsergeymirzoev@gmail.com}}}
\title{Оценка бессрочных опционов для случайных процессов со скачками}
\affil{Факультет Вычислительной Математики и Кибернетики, Московский государственный университет им. М.В.Ломоносова}

\makeatletter
% Roman numbers support
\newcommand*{\rom}[1]{\expandafter\@slowromancap\romannumeral #1@}

% Derivative at point support
\newcommand{\at}[2][]{#1|_{#2}}

% Expectation support
\newcommand{\expect}{\operatorname{E}\expectarg}
\DeclarePairedDelimiterX{\expectarg}[1]{[}{]}{%
  \ifnum\currentgrouptype=16 \else\begingroup\fi
  \activatebar#1
  \ifnum\currentgrouptype=16 \else\endgroup\fi
}

\newcommand{\innermid}{\nonscript\;\delimsize\vert\nonscript\;}
\newcommand{\activatebar}{%
  \begingroup\lccode`\~=`\|
  \lowercase{\endgroup\let~}\innermid 
  \mathcode`|=\string"8000
}

\renewcommand\qedsymbol{$\blacksquare$}

\makeatother

\begin{document}

\begin{titlepage}
\centering

\textsc{Дипломная работа}

\vspace{\stretch{1}}

{\LARGE\bfseries Оценка бессрочных опционов для случайных процессов со скачками\\}
\rule{3in}{0.4pt}

\vspace{\stretch{1}}

Мирзоев С.Е.\\
Московский государственный университет им. М.В.Ломоносова\\
Факультет Вычислительной Математики и Кибернетики\\
Октябрь 2021

\vspace{\stretch{1}}

{\small
Научный руководитель: Белянкин Г.А.\\
Со-руководитель: Морозов В.В.}

\vspace*{\stretch{1}}

\end{titlepage}
\thispagestyle{empty}

\newpage
\tableofcontents

\newpage

%%%%%%%%%%%%%%%%%%%%%%%%%%%%%%%%%%%%
%%%%%%%%%%%% Вступление %%%%%%%%%%%%
%%%%%%%%%%%%%%%%%%%%%%%%%%%%%%%%%%%%
\section{Вступление}

В статье Гербера и Шиу была сделана оценка бесконечного пут-опциона, в данной статье аналогичный результат будет повторён для колл-опциона. В указанной статье рассматриваются две модели, в которых логарифм цены актива представляет собой сдвинутый сложный пуассоновский процесс. Явные результаты получены для цен и оптимальных стратегий исполнения определенных бессрочных американских опционов на актив, в частности для бессрочного пут-опциона. В первой модели, в которой скачки цены актива направлены вверх, результаты получаются с использованием процесса - мартингала и условия гладкого сшивания. Во второй модели, в которой скачки идут вниз, показывается, что значение стратегии, соответствующее постоянной границе исполнения опциона удовлетворяет определенному уравнению обновления. Получаем оптимальную стратегию исполнения, удовлетворяющую условию непрерывного сшивания. Кроме того, одна и та же модель может быть использована для определения цены на определенные параметры отмены. Наконец, было показано, как классическая модель геометрического броуновского движения может быть использована в качестве предела, а также, как она может быть встроена в обе модели.

%%%%%%%%%%%%%%%%%%%%%%%%%%%%%%%%%%%%
%%%%%%%%%%%%% Введение %%%%%%%%%%%%%
%%%%%%%%%%%%%%%%%%%%%%%%%%%%%%%%%%%%

\section{Введение}

Пусть S(t) - цена актива, например, акции, в момент времени $t \ge 0$. Мы предполагаем, что рынок риск-нейтрален, следовательно, стоимость ценной бумаги - это математическое ожидание дисконтированных будущих платежей. В предположении, что актив не выплачивает никаких дивидендов и что безрисковая мгновенная процентная ставка $r$ является положительной константой, случайный процесс $\{e^{-rt} S(t)\}$, представляющий собой дисконтированную цену актива  является мартингалом.

Пусть $U(t) = ln S(t), t \ge 0$. Мы предполагаем, что $\{U(t), t \ge 0\}$ - процесс с независимыми и стационарными приращениями, с начальным значением $U(0) = u$. Классическая модель цены акций является частным случаем, в котором $\{U(t)\}$ представляет собой винеровский процесс со сдвигом $\mu = r - \frac{\sigma ^ {2}}{2}$ и бесконечно малой дисперсией $\sigma ^ {2}$. 

В данной работе будут рассмотрены две модели:

\textbf{Модель \rom{1}:}
\begin{equation}\label{eq:model1_definition}
    U(t) = u - ct + Z(t)
\end{equation}
\textbf{Модель \rom{2}\footnote{Заметим, что модель \rom{2} напоминает модель, которая используется для процесса избытка в классической теории риска.}:} 
\begin{equation}\label{eq:model2_definition}
    U(t) = u + ct - Z(t)
\end{equation}

В обоих случаях $c > 0$ и является постоянной величиной, а $\{Z(t)\}$ представляет собой сложный пуассоновский процесс, определяемый параметром $\lambda > 0$ и распределением величин скачка. Заметим также, что скачки положительны.

\label{sec:positivityOfJumsAssumption} Для упрощения обозначений предположим, что распределения величин скачка
непрерывны с плотностью вероятности $p(x), x \ge 0$. 

\begin{lemma}[Отсутствие арбитража]\label{thm:no_arbitrage_theoreme}
В модели \rom{2} мы делаем дополнительное предположение, что $c > r$ иначе возникает возможность арбитража. 
\end{lemma}
\begin{proof}
Если бы было допустимо $c \le r$, мы бы имели $S(t) \le S(0)e^{rt}$ со строгим неравенством после первого скачка, т.к. скачки положительны. Следовательно, путем короткой продажи (Подробнее: \ref{def:short_position}) актива и инвестирования выручки $S(0)$ по безрисковой процентной ставке $r$ мы могли бы получить безрисковую прибыль.

Действительно, пусть в момент $0$ инвестор берёт актив в короткую позицию, сразу её продаёт и сумму $S(0)$ кладёт на депозит в банк под ставку $r$. Далее он ждёт, когда произойдет первый скачок актива и выполнится $e^{rt} S(0) > S(t)$. Наконец, инвестор забирает деньги с банковского депозита, покупает актив, возвращает его брокеру и имеет безрисковую прибыль $e^{rt}S(0) - S(t)$.
\end{proof}

Наша главная цель - оценить бессрочный американский
опцион с функцией выплаты $P(s)$. Хотя указание \textit{американский} не является обязательным, оно добавляется, чтобы подчеркнуть, что опцион может быть реализован в любой момент.

Если такой опцион будет реализован в момент времени $T$, прибыль составит $P(S(T))$. Ограничимся опционами с функциями выплаты, для которых априори ясно, что оптимальными стратегиями исполнения будут моменты остановки вида:

\begin{equation}\label{eq:optimal_excersize}
T_L = \inf\limits_{S(t) < L}{t},
\end{equation}

где граница исполнения опциона $L$ является положительной константой. Это, в частности, относится к бессрочному опциону \textbf{пут}, где 
\begin{equation}\label{eq:payoff_function}
P(s) = \max(K - s, 0) = (K - s)_+,
\end{equation}
где $K$, обозначает цену исполнения. Проблема состоит в
том, чтобы сначала найти стоимость стратегии $T_L$,

\begin{equation}\label{eq:strategy_cost}
V(s; L) = \expect*{e^{-rT_L} P(S(T_L)) | S(0) = s}, s \ge L,
\end{equation}

а затем определить оптимальное значение $\widetilde{L}$, которое
максимизирует $V(s; L)$. Такое значение $\widetilde{L}$ - оптимальная граница опциона. Цена опциона равна:

\begin{equation}\label{eq:option_price}
    \begin{cases}
      V(S(0); \widetilde{L}), \text{ при } S(0) \ge \widetilde{L}\\
      P(S(0)), \text{ при } S(0) < \widetilde{L}
    \end{cases}\,.
\end{equation}

%%%%%%%%%%%%%%%%%%%%%%%%%%%%%%%%%%%%
%%%%%% Решение для модели 1 %%%%%%%%
%%%%%%%%%%%%%%%%%%%%%%%%%%%%%%%%%%%%
\section{Решение для модели \rom{1}}

Поскольку траектории процесса $\{S(t)\}$ не имеют скачков вниз, точная нижняя грань \eqref{eq:optimal_excersize} достигается на непрерывном участке траектории, а значит $S(T_L) = L$, и, следовательно, \eqref{eq:strategy_cost} получает вид:

\begin{equation}\label{eq:strategy_c1}
V(s; L) = \expect*{e^{-rT_L} | S(0) = s} P(L), s \ge L,
\end{equation}

В \eqref{eq:strategy_c1} остается определить мат.ожидание. Для этого решим вспомогательную задачу - найдём все такие $\xi$, при которых случайный процесс $\{e^{-rt}S(t)^{\xi}\}$ является мартингалом.
\begin{theorem}[Мартингальное условие для $\xi$]\label{thm:martingale_cond_for_xi}
Пусть для некоторого $\xi$, случайный процесс $\{e^{-rt}S(t)^{\xi}\}$ является мартингалом. 

Так как $e^{-rt}S(t)^{\xi} = e^{-rt + \xi U(t)}$ и $\{U(t), t \ge 0\}$ - процесс с независимыми и стационарными приращениями, условие мартингальности для $\xi$ при $t = 1$ \footnote{выбор конкретного момента реализации процесса $t=1$ не сужает общности решаемой задачи} выглядит как:
\begin{equation}\label{eq:martingale_cond_for_xi}
e^{-r} \expect*{e^{\xi U(1)}|U(0) = u} = e ^ {\xi u}
\end{equation}

Учитывая определение модели \rom{1} \eqref{eq:model1_definition}:
\begin{equation}\label{eq:martingale_cond_for_xi_alt}
e^{-r - c\xi} \expect*{e^{\xi Z(1)}|U(0) = u} = 1
\end{equation}

Что приводит к:
\begin{equation}\label{eq:martingale_cond_for_xi_log}
\lambda \left[ \int_{0}^{+\infty} e^{\xi x} p(x) \,dx - 1  \right] - r - c \xi
 = 0
\end{equation}

\end{theorem}
\begin{proof}
Условие мартингальности для $\{e^{-rt} S(t)^{\xi}\}$ :
\begin{equation*}
\expect*{e^{-rt} S(t)^{\xi}|S(0) = s} = s^{\xi}
\end{equation*}
Учитывая определение процесса $\{U(t) = \ln{S(t)}\}$:
\begin{equation*}
\expect*{e^{-rt + \xi U(t)}|U(0) = u} = e^{\xi u}
\end{equation*}
Подставим $t = 1$. Как отмечалось выше, рассмотрение конкретного момента реализации процесса не сужает общности решаемой задачи:
\begin{equation*}\label{eq:thm2_eq1}
e^{-r}\expect*{e^{\xi U(1)}} = e^{\xi u}
\end{equation*}

Из формулировки первой модели \eqref{eq:model1_definition}, уравнение выше принимает вид:
\begin{equation*}
e^{-r}\expect*{e^{\xi(u - c + Z(1))}} = e^{\xi u}
\end{equation*}
Или:
\begin{equation*}
e^{-r - c \xi} \expect*{e^{\xi Z(1)}} = 1
\end{equation*}

Заметим, что математическое ожидание в выражении выше представляет собой производящую функцию моментов обобщенного пуассоновского процесса (Подробнее: \ref{def:compound_poisson}):

\begin{equation}\label{eq:thm2_eq2}
e^{-r - c \xi} M_{Z(1)}(\xi) = 1
\end{equation}

Для её расчёта можно применить следующую лемму.

\begin{lemma}[Производящая функция моментов обобщенного пуассоновского процесса]\label{thm:thm2_moment_generating_function}
Обозначим производящую функцию моментов случайной величины $X$:
\begin{equation*}
    M_X(\alpha) = \expect{e^{\alpha X}}
\end{equation*}
Тогда для обобщенного Пуассоновского процесса $\{X(t)\}$.
\begin{equation}\label{eq:moment_generating_for_poisson}
     M_{X(t)}(\alpha) = e^{\lambda t \left(M_{Y_1}(\alpha) - 1\right)}
\end{equation}
\end{lemma}
\begin{proof}
Распишем производящую функцию моментов для обобщенного Пуассоновского процесса:
\begin{equation*}
     M_{X_t}(\alpha) = \expect[\big]{e^{\alpha X_t}} = \expect[\big]{e^{\alpha \sum_{i=1}^{N_t} Y_i}}
\end{equation*}

Также заметим, что распределение $N_t$ нам известно:
\begin{equation}\label{eq:thm2_moment_generating_function_eq1}
     \expect[\big]{e^{\alpha \sum_{i=1}^{N_t} Y_i}} = \sum_{m=0}^{\infty} \expect[\big]{e^{\alpha \sum_{i=1}^{m} Y_i}} P(N_t = m) = \sum_{m=0}^{\infty} \expect[\big]{e^{\left(\alpha \sum_{i=1}^{m} Y_i\right)}} \frac{(\lambda t)^{m} e^{-\lambda t}}{m!}
\end{equation}

Так как случайные величины $Y_i$ независимы и одинаково распределены, верно:
\begin{equation}\label{eq:thm2_moment_generating_function_eq2}
     \expect[\big]{e^{\alpha \sum_{i=1}^{m} Y_i}} = \expect[\big]{\prod_{i=1}^{m} e^{\alpha Y_i}} = \prod_{i=1}^{m}  \expect[\big]{e^{\alpha Y_i}} = \prod_{i=1}^{m}  \expect[\big]{e^{\alpha Y_1}} = \left( \expect[\big]{e^{\alpha Y_1}} \right)^{m}
\end{equation}

Далее необходимое получается при подстановке \eqref{eq:thm2_moment_generating_function_eq1} и \eqref{eq:thm2_moment_generating_function_eq2} в цепочку уравнений:
\begin{equation*}
\begin{split}
    \sum_{m=0}^{\infty} \expect{e^{\left(\alpha \sum_{i=1}^{m} Y_i\right)}} \frac{(\lambda t)^{m} e^{-\lambda t}}{m!} &= \sum_{m=0}^{\infty} \left(\expect{e^{\alpha Y_1}}\right)^{m} \frac{(\lambda t)^{m} e^{-\lambda t}}{m!} =\\
    = \sum_{m=0}^{\infty} (M_{Y_1} (\alpha))^{m} \frac{(\lambda t)^{m} e^{-\lambda t}}{m!} &= \sum_{m=0}^{\infty} \frac{(M_{Y_1} (\alpha) \lambda t)^{m}}{m!} e^{-\lambda t} =\\
    = e^{\left(M_{Y_1} (\alpha) \lambda t\right)} e^{-\lambda t} &= e^{\lambda t \left(M_{Y_1}(\alpha) - 1\right)}
\end{split}
\end{equation*}
\end{proof}

Теперь прологарифмируем обе части \eqref{eq:thm2_eq2} и подставим результат леммы  \ref{thm:thm2_moment_generating_function}:
\begin{equation*}
\ln{M_{Z(1)}(\xi)} - r - c\xi = 0
\end{equation*}
\begin{equation*}
\ln{e^{\lambda (M_{Y_1}(\xi) - 1)}} - r - c\xi = 0
\end{equation*}

Упростив полученное:
\begin{equation*}
\lambda \left(M_{Y_1}(\xi) - 1\right) - r - c\xi = 0
\end{equation*}

Подставим выражения для мат.ожидания скачка получим требуемое:
\begin{equation*}
\lambda \left(\int_{0}^{+\infty} e^{\xi x} p(x) \,dx - 1\right) - r - c\xi = 0
\end{equation*}
\end{proof}

Далее сформулируем утверждение о виде функции в левой части уравнения \eqref{eq:martingale_cond_for_xi_log}.
\begin{lemma}[Строгая выпуклость]\label{thm:strict_convexity_m1}
Выражение в левой части уравнения \eqref{eq:martingale_cond_for_xi_log} является строго выпуклой функцией по $\xi$.
\end{lemma}
\begin{proof}
Рассмотрим вторую производную данной функции:
\begin{equation*}
\begin{split}
     \pdv[2]{}{\xi} \left[\lambda \left(\int_{0}^{+\infty} e^{\xi x} p(x) \,dx - 1\right) - r - c\xi \right] &= \pdv{}{\xi} \left[\lambda \left(\int_{0}^{+\infty} e^{\xi x} p(x) x \,dx \right) - c \right] =\\
     = \lambda \left(\int_{0}^{+\infty} e^{\xi x} p(x) x^{2} \,dx \right) &= \lambda \expect*{e^{\xi x} x^2}
\end{split}
\end{equation*}

Так как $\lambda > 0$, $x^2 \ge 0$, $e^{\xi x} > 0$, $p(x > 0) \neq 0$, выражение $\lambda \expect*{e^{\xi x} x^2}$ строго положительно, как произведение положительной величины и мат. ожидания ненулевой неотрицательной случайной величины. Итак, вторая производная исходной функции строго положительна.

\end{proof}

Учитывая утверждение выше, уравнение \eqref{eq:martingale_cond_for_xi_log} имеет не более двух решений. На самом деле, оно имеет ровно два решения.
\begin{lemma}[О числе решений уравнения \eqref{eq:martingale_cond_for_xi_log}]\label{thm:solution_for_mart_cond_m1}
Уравнение \eqref{eq:martingale_cond_for_xi_log} имеет ровно два решения. \\
Первое: $\xi_1 = 1$, \\
Второе: $\xi_2 = -R < 0$, для некоторого $R > 0$
\end{lemma}
\begin{proof}
Обозначим функцию в левой части уравнения \eqref{eq:martingale_cond_for_xi_log} через $f(\xi)$:
\begin{equation*}
f(\xi) = \lambda \left(\int_{0}^{+\infty} e^{\xi x} p(x) \,dx - 1\right) - r - c\xi
\end{equation*}

Так как ${e^{-rt} S(T), t \ge 0}$ является мартингалом и в то же время является частным случаем процесса  $\{e^{-rt}S(t)^{\xi}\}$ при $\xi_1 = 1$, естественным образом получаем первое решение.

Второе решение будет отрицательным, то есть для некоторого $R > 0$ будет иметь вид: $\xi_2 = -R < 0$.
Это следует из того, что:

\begin{equation*}
f(0) = \lambda \left(\int_{0}^{+\infty} e^{0} p(x) \,dx - 1\right) - r = \lambda (1 - 1) - r = -r
\end{equation*}

\begin{equation*}
\begin{split}
\lim_{\xi\to-\infty} f(\xi) &= \lim_{\xi\to-\infty} \left[ \lambda \left(\int_{0}^{+\infty} e^{\xi x} p(x) \,dx - 1\right) - r - c\xi \right] =\\
= - \lambda - r - c \xi &= +\infty
\end{split}
\end{equation*}

А так как выражение в левой части - непрерывная функция от $\xi$, она принимает все значения лежащие в $[-R, +\infty)$ на $\xi \in (-\infty, 0]$ по теореме Больцано-Коши о промежуточном значении.
\end{proof}

Отметим, что положительный мартингал $\{e^{-rt} S(t)^{-R}\}$ ограничен сверху константой $L^{-R}$ при $t < T_L$. 

Данный факт следует из определения $T_L$ \eqref{eq:optimal_excersize}:
\begin{equation*}
S(t) \ge L, \forall t < T_L
\end{equation*}
Значит:
\begin{equation*}
e^{-rt}S(t)^{-R} \le e^{-rt}L^{-R} < L^{-R}
\end{equation*}

Следовательно, мы можем применить теорему о преобразовании свободного выбора \cite{bib:Shiryaev}, чтобы сделать следующий вывод:
\begin{theorem}
\begin{equation}\label{eq:opt_sampl_thm_application_to_mart}
s^{-R} = \expect*{e^{-rT_L} | S(0) = s} L^{-R}
\end{equation}
\end{theorem}
\begin{proof}
% https://almostsuremath.com/2009/12/20/optional-sampling/
Итак, теорема о преобразовании свободного выбора (\ref{def:OptSamplTheorem}) для мартингала сводится к следующему:

Если ${X}$ - мартингал, то $\expect*{X_{\tau} | \mathcal{F}_\sigma} = X_\sigma$.

В нашем случае: $\sigma = 0$, $\mathcal{F}_\sigma = \{S(0) = s\}$, а $X_{\tau} = e^{-r\tau} S(\tau)^{-R}$. Тогда:
\begin{equation*}
    \expect*{e^{-rT_L} S(T_L)^{-R} | S(0) = s} = X_\sigma = X_0 = S(0)^{-R} = s^{-r}
\end{equation*}

Вспомним определение $T_L$ \eqref{eq:optimal_excersize} и подставим $S(T_L) = L$:
\begin{equation}\label{eq:second_part_of_opt_sampl_thm_application_to_mart}
     \expect*{e^{-rT_L} S(T_L)^{-R} | S(0) = s} = \expect*{e^{-rT_L} | S(0) = s} L^{-R}
\end{equation}

Приходим к требуемому равенству.
\end{proof}

Соединяя полученные в \eqref{eq:opt_sampl_thm_application_to_mart} и \eqref{eq:strategy_c1} результаты получаем:

\begin{equation}\label{eq:strategy_c1_simplified}
V(s; L) = \left(\frac{L}{s}\right)^{R} P(L), s \ge L.
\end{equation}

Осталось найти $\widetilde{L}$, то есть значение $L$, максимизирующее $L^{R} P(L)$. Тогда условие первого порядка имеет вид:

\begin{equation*}
    \pdv{(L^{R} P(L))}{L} = R L^{R - 1} P(L) + L^R P'(L) = 0
\end{equation*}
\begin{equation}\label{eq:first_order_rule_m1}
    \frac{R}{\widetilde{L}} P(\widetilde{L}) + P'(\widetilde{L}) = 0
\end{equation}

Более того, используя \eqref{eq:strategy_c1_simplified} и \eqref{eq:first_order_rule_m1} можно увидеть, что:

\begin{equation}\label{eq:smooth_pasting_condition_m1}
    \pdv{V(s; \widetilde{L})}{s}\at[\big]{s=\widetilde{L}} = P'(\widetilde{L})
\end{equation}

Это означает, что функции $P(s), s \le \widetilde{L}$ и $V(s; \widetilde{L}), s \ge \widetilde{L}$, имеют гладкое соединение в точке $s = \widetilde{L}$. Этот тип условия оптимальности также называется \textit{гладким сшиванием} или \textit{условием высокого контакта}.

В следующем разделе представлен основной вклад этой статьи, оценивающий бессрочные опционы на активы, цены на которые могут перескакивать через оптимальную границу опциона-исполнения.

%%%%%%%%%%%%%%%%%%%%%%%%%%%%%%%%%%%%
%%%%%%%%%%%% Заключение %%%%%%%%%%%%
%%%%%%%%%%%%%%%%%%%%%%%%%%%%%%%%%%%%
\section{Заключение}

%%%%%%%%%%%%%%%%%%%%%%%%%%%%%%%%%%%%
%%%%%%%%%%%% Литература %%%%%%%%%%%%
%%%%%%%%%%%%%%%%%%%%%%%%%%%%%%%%%%%%
\section{Сопутствующая литература}

\begin{thebibliography}{3}
\bibitem{bib:GerberShiu}
Hans U. Gerber and Elias S.W. Shiu \textbf{"Pricing Perpetual Options for Jump Processes"} // --- North American Actuarial Journal, 1998 --- vol. 2, issue 3, 108-109 ---
\bibitem{bib:Shiryaev}
А. Н. Ширяев, \textbf{"О мартингальных методах в задачах о пересечении границ броуновским движением"} // --- Совр. пробл. матем., 2007 --- выпуск 8, 3–78 ---
\end{thebibliography}

\appendix
% https://math.stackexchange.com/questions/1090696/moment-generating-function-of-a-compound-poisson-process
%%%%%%%%%%%%%%%%%%%%%%%%%%%%%%%%%%%%
%%%%%% Приложение: Определения %%%%%
%%%%%%%%%%%%%%%%%%%%%%%%%%%%%%%%%%%%
\section{Определения}
\begin{definition}[Обобщенный пуассоновский процесс]
    \label{def:compound_poisson}
    Процесс $\{X_t, t \ge 0\}$ называется \textbf{обобщенным процессом Пуассона}, если мы можем записать его как:
    
    \begin{equation}
        X_t = \sum_{i=1}^{N_t} Y_i,
    \end{equation}
    
    где $\{N_t, t \ge 0\}$ - процесс Пуассона, а $\{Y_n\}_{n \ge 1}$ - семейство случайных независимых v. с тем же распределением, которые также независимы от $\{N_t, t \ge 0\}$.
\end{definition}
\begin{definition}[Шорт, короткая позиция]
    \label{def:short_position}
    \textbf{Продажа без покрытия или торговля в шорт} - продажа ценных бумаг, товаров или валюты, которыми торговец на момент продажи не владеет. Такая операция возможна, если условия контракта предусматривают его исполнение (поставку) через некоторое время или при маржинальной торговле, когда разрешено продавать взятый у брокера в кредит товар с предполагаемой последующей покупкой аналогичного товара и возврата кредита в натуральном (товарном) виде. Торговец надеется, что цена упадёт и он сможет дешевле выкупить ранее проданный товар. Этот механизм обеспечивает возможность получать прибыль при снижении цен.
\end{definition}
\begin{definition}[Теорема о преобразовании свободного выбора]\label{def:OptSamplTheorem}
    Пусть ($\Omega$, $\mathcal{F}$,($\mathcal{F}_t)_{t \ge 0}$, P) – вероятностное пространство с непрерывной справа фильтрацией ($\mathcal{F}_t)_{t \ge 0}$ (т.е. ($\mathcal{F}_t)^{+} \equiv \cap_{u > t} \mathcal{F}_u = \mathcal{F}_t$ для
любого $t \ge 0$), пополненной множествами P-меры нуль из $\mathcal{F}$.

Все случайные процессы $X = (X_t)_{t \ge 0}$, которые мы будем рассматривать на ($\Omega$, $\mathcal{F}$,($\mathcal{F}_t)_{t \ge 0}$, P), предполагаются согласованными с ($\mathcal{F}_t)_{t \ge 0}$ (т.е. $X_t$ является $F_t$-измеримым для любого $t \ge 0$) и имеющими непрерывные справа траектории.

1) Пусть $X = (X_t)_{t \ge 0}$ – субмартингал, а $\sigma$ и $\tau$ – два момента остановки, причем момент $\tau$ ограничен ($\tau \le T$). Тогда случайная величина $X_\tau$ интегрируема и P-п.н.:

\begin{equation}\label{eq:os_thm_eq1}
    \expect*{X_\tau | \mathcal{F}_\sigma} \ge X_{\tau \wedge \sigma},
    \textit{где } \mathcal{F}_\infty = \sigma(\cap_{t \ge 0} \mathcal{F}_t)
\end{equation}

Пусть $M = (M_t)_{t \ge 0}$ – мартингал, а $\sigma$ и $\tau$ – два момента остановки, причем $\tau$ ограничен ($\tau \le T$). Тогда случайная величина $M_\tau$ интегрируема и P-п.н.

\begin{equation}\label{eq:os_thm_eq2}
    \expect*{M_\tau | \mathcal{F}_\sigma} = M_{\tau \wedge \sigma},
    \textit{где } \mathcal{F}_\infty = \sigma(\cap_{t \ge 0} \mathcal{F}_t)
\end{equation}

% 2) Свойства \eqref{eq:thm2_moment_generating_function_eq1} и \eqref{eq:thm2_moment_generating_function_eq2} выполняются также и для неограниченных  , если семейства (X + t)t>0 и (|Mt|)t>0 равно-
% мерно интегрируемы (т.е. supt>0 E [X

% +
% t
% I(X
% +
% t > c)] → 0 и

% supt>0 E [|Mt|I(|Mt| > c)] → 0 при c → ∞).
\end{definition}

\end{document}
