\documentclass[a4paper,12pt]{article}

\usepackage{hyperref}
\usepackage[affil-it]{authblk}
\usepackage[rgb]{xcolor}
\hypersetup{				% Гиперссылки
    colorlinks=true,       	% false: ссылки в рамках
	urlcolor=blue           % на URL
}

%  Русский язык
\usepackage[T2A]{fontenc}			% кодировка
			% кодировка исходного текста
\usepackage[english,russian]{babel}	% локализация и переносы
\usepackage[utf8]{inputenc}

% Математика
\usepackage{amsmath,amsfonts,amssymb,amsthm,mathtools}


\usepackage{wasysym}

%Заголовок
\author{Мирзоев С.Е.%
\thanks{E-mail: \texttt{mrsergeymirzoev@gmail.com}}}
\title{Поиск оптимальной стратегии перестрахования в кредитном страховании}
\affil{Факультет Вычислительной Математики и Кибернетики, Московский государственный университет им. М.В.Ломоносова}

\makeatother

\begin{document}

\begin{titlepage}
\centering

\textsc{Курсовая работа}

\vspace{\stretch{1}}

{\LARGE\bfseries Поиск оптимальной стратегии перестрахования в кредитном страховании\\}
\rule{3in}{0.4pt}

\vspace{\stretch{1}}

Мирзоев С.Е.\\
Московский государственный университет им. М.В.Ломоносова\\
Факультет Вычислительной Математики и Кибернетики\\
Май 20, 2021

\vspace{\stretch{1}}

{\small
Научный руководитель: Белянкин Г.А.\\
Со-руководитель: Палкин А.В.}

\vspace*{\stretch{1}}

\end{titlepage}
\thispagestyle{empty}

\newpage
\tableofcontents

\newpage

\section{Вступление}
Страхование торговых кредитов (СТК) - это инструмент управления рисками, обычно используемый поставщиками как гарантия от неплатежей покупателей услуг кредитования. Контракты СТК могут быть либо аннулируемыми (страховщик имеет право по своему усмотрению отменить эту гарантию в течение страхового периода), либо не подлежащими аннулированию (условия не могут быть пересмотрены в течение страхового периода). В настоящей работе определены две роли СТК: роль сглаживания денежных потоков поставщика и роль мониторинга (отслеживание дальнейшей кредитоспособности покупателя после заключения контракта, что позволяет поставщику принимать эффективные операционные решения относительно того, следует ли продавать товары покупателю кредита). Мы далее исследуем, какие контракты  лучшим образом облегчают эти две роли СТК, моделируя стратегическое взаимодействие между страховщиком и поставщиком. Контракты, не подлежащие отмене, полагаются на франшизу для реализации обеих ролей, что может привести к конфликту: высокая франшиза препятствует роли сглаживания, в то время как низкая франшиза ослабляет роль мониторинга. При расторжении договоров действие страховщика по расторжению гарантирует, что полученная информация будет отражена в решении поставщика о доставке. Таким образом, страховщик имеет достаточные стимулы для выполнения своей функции мониторинга, не прибегая к высокой франшизе. Несмотря на это преимущество, мы считаем, что страховщик может использовать опцию отмены слишком агрессивно; таким образом, это делает предпочтительными не подлежащие расторжению контракты, особенно когда опцион поставщика непривлекателен, а затраты страховщика на мониторинг низки. Не подлежащие отмене контракты также более привлекательны, нежели когда полученная информация поддается проверке, чем когда она не поддается проверке.

\section{Введение}
Торговый кредит возникает, когда поставщик позволяет покупателю отсрочить оплату уже поставленных товаров или услуг. Существует большой объем работ, которые свидетельствуют о том, что торговый кредит значительно облегчает торговлю (Giannetti et al., 2011). Однако поставщик, предоставляющий кредит, также подвержен риску неплатежа, то есть покупатель может существенно задержать платеж или вообще не оплатить его. Такие дефолты могут создать серьезные финансовые трудности для поставщиков, особенно среди малых и средних предприятий (МСП).


Чтобы защитить себя от таких негативных событий, поставщики часто приобретают к страхованию торговых кредитов (СТК), которое позволяет застрахованной стороне возместить убытки, возникшие в результате дефолта покупателя. При принятии решения о том, следует ли предоставлять товары кредитному покупателю, поставщик уравновешивает выгоду от торговой сделки с рисками, связанными с отсутствием платежей. СТК помогает склонить баланс в пользу выдачи кредита для клиента и тем самым улучшает торговлю. СТК может использоваться для страхования одной сделки между поставщиком и покупателем, на которой и фокусируется данная статья, или всей торговли между поставщиком и заранее определенным набором покупателей на фиксированный срок (обычно на год). СТК впервые стал популярным в Европе, однако теперь он имеет большое глобальное распространение. По состоянию на 2016 год СТК охватил более 2,3 трлн евро по всему миру (International Credit Insurance and Surety Association 2017).


СТК в некоторых отношениях напоминает традиционное страхование, в других же отличается. В частности, он напоминает остальные формы страхования тем, что предназначен для гарантирования выплаты застрахованной стороне (поставщику) при определенных условиях. Однако, в отличие от договоров, используемых в остальных условиях страхования, договор СТК часто позволяет страховщику полностью или частично отменить покрытие в любое время до поставки товара. Это достигается путем сначала определения кредитного лимита (покрытия) на определенный период времени, а затем в течение самого этого периода, позволяя страховщику отозвать или изменить эти лимиты, если страховщик поймет, что произошло значительное изменение кредитного риска (Jones 2010, Association of British Insurers 2016).

Страховщики довольно часто проявляют гибкость, чтобы отменить лимит, когда считают, что риск неплатежеспособности снизился. Только за первые несколько месяцев 2009 года аннулированные кредитные линии СТК составили 75 миллиардов евро (International Financial Consulting 2012). Как только СТК отзывается для своих покупателей, поставщики, как правило, прекращают доставку в кредит. Эта динамика, естественно, подрывает способность обеих сторон торговать (Kollewe 2009). По оценкам, от 5\% до 9\% общего падения мирового экспорта во время недавнего финансового кризиса можно объяснить отказом от страхования торговых кредитов (van der Veer 2015). Необоснованная отмена негативно влияет на эффективность решения о доставке и приводит к экономическим потерям. Таким образом, использование варианта отмены часто является спорным из-за его значительных операционных и финансовых последствий. По словам Билла Гримси, исполнительного директора британского ритейлера товаров для дома Focus DIY, кредитные страховщики-это “друзья в хорошую погоду, которые не вдаются в детали, принимают односторонние решения в кратчайшие сроки и ставят под угрозу будущее бизнеса” (Stacey 2009).

Альтернативный подход к заключению контрактов на страхование кредитов как ответ на опасения по поводу отмены воплощен в неотменяемом покрытии. Как следует из названия, кредитные лимиты в неотменяемом покрытии не могут быть изменены в течение определенного периода времени (AIG 2013, Ассоциация британских страховщиков 2016). Учитывая высокие ставки, относительные достоинства двух подходов к заключению контрактов горячо обсуждаются в сообществе практиков (Aitken 2013).

Основываясь на приведенном выше обсуждении, мы исследуем следующие вопросы в этой работе: 1) Какую роль играет СТК в цепочке поставок? 2) Каковы относительные преимущества расторгаемых и не расторгаемых контрактов и какими принципами следует руководствоваться при выборе формы контракта?

Чтобы ответить на эти вопросы, мы разрабатываем теоретико-игровую модель, которая отражает стратегические взаимодействия между поставщиком и страховщиком для одной сделки. Рисковый покупатель размещает кредитный заказ у поставщика, который берет на себя финансовые расходы, столкнувшись с нехваткой денежных средств. Поставщик покупает СТК у страховщика, чтобы защитить себя от риска неплатежа. После этого страховщик может предпринять дорогостоящие, но не поддающиеся проверке усилия по мониторингу, чтобы получить обновленную информацию о риске дефолта покупателя. Эта характерная особенность модели отражает тот факт, что СТК покрывает риск дефолта покупателя, который не является внутренним для страховой фирмы (поставщика) и может изменяться во времени. На практике страховщик часто обладает превосходными возможностями для сбора и анализа информации о риске дефолта покупателя по различным каналам, таким как тщательный мониторинг общедоступных источников информации (например, финансовой отчетности) и непубличных источников, таких как посещение сайтов, а также отслеживание истории платежей фирмы-покупателя (Jones 2010, Amiti and Weinstein 2011, Ассоциация британских страховщиков 2016, Euler Hermes 2018). Полученные данные о рынке, которые могут отражать ухудшение политического или географического риска (2012) или ухудшение финансового положения покупателя (Birchall 2008), оказались эффективными в прогнозировании дефолта по платежам (Kallberg and Udell 2003, Cascino et al. 2014). На основе обновленной оценки риска покупателя, предоставленной страховщиком, поставщик может пересмотреть свое решение об отправке заказа.


Сравнивая централизованный контрольный показатель со сценарием, в котором страхование недоступно, мы количественно оцениваем две роли, которые выполняет СТК. Во - первых, с финансовой стороны, СТК сглаживает денежный поток поставщика при различных возможных реализациях риска покупателя и тем самым смягчает неблагоприятные финансовые последствия неисполнения платежей. Мы называем это ролью сглаживания (денежного потока) СТК. Во-вторых, в отличие от других параметров страхования, СТК также является информационной службой, которая облегчает принятие оперативных решений. В частности, получив доступ к информации, которую страховщик, возможно, собрал относительно развивающейся кредитоспособности покупателя, поставщик может принимать более эффективные операционные (транспортные) решения. Направляя решение о доставке в сторону эффективности, эта контрольная роль СТК создает операционную ценность, которая выходит за рамки чисто финансовых соображений. Выполнение функции мониторинга СТК требует, чтобы у страховщика был стимул получать обновленную информацию и чтобы поставщик принимал эффективные решения о доставке на основе обновленной информации. Эти два вопроса стимулирования являются решающими факторами, определяющими оптимальную форму контрактов на ОТК.

Мы в основном фокусируемся на двух формах контрактов СТК: не подлежащие отмене и аннулируемые контракты. В соответствии с договором не подлежащим аннулированию, который аналогичен стандартному договору в других условиях страхования, кредитный платеж от покупателя за вычетом франшизы гарантируется страховщиком в обмен на премию. Однако аннулируемый контракт дает страховщику возможность отменить покрытие до того, как заказ будет отправлен. Мы считаем, что не подлежащий отмене контракт с франшизой ограничен в своей способности выполнять как функции сглаживания, так и функции мониторинга СТК. С одной стороны, низкая франшиза не удерживает поставщика от доставки, даже когда кредитоспособность покупателя ухудшается. Такая неэффективная реакция на обновленную информацию снижает стимул страховщика к мониторингу, тем самым ослабляя контрольную роль СТК. С другой стороны, высокая франшиза имеет два возможных недостатка: во-первых, она может подвергнуть поставщика более высоким затратам на финансирование, ставя под угрозу роль сглаживания; во-вторых, она также может препятствовать мониторингу, снижая подверженность страховщика кредитным потерям. Таким образом, не подлежащие расторжению контракты могут возместить полную стоимость СТК только в том случае, если поставщик финансово обеспечен, а затраты страховщика на мониторинг невелики.


В отличие от этого, при расторжении контрактов возможность страховщика отменить покрытие позволяет включить полученную им информацию о риске покупателя, чтобы надлежащим образом повлиять на решение поставщика о доставке. В частности, по сравнению с договорами, не подлежащими отмене, страховщик теперь более мотивирован инвестировать в дорогостоящий мониторинг, поскольку, узнав, что покупатель чрезмерно рискован, он может удержать поставщика от отгрузки, воспользовавшись возможностью отмены, не прибегая к высокой франшизе. Эта динамика объясняет, почему аннулируемые договоры страхования долгое время преобладали в отрасли СТК.


Несмотря на это преимущество, мы обнаруживаем, что, поскольку страховщик не получает прямой выгоды от сделки после подписания договора страхования, страховщик может чрезмерно отменить, т.е. отменить покрытие поставщика, даже если оно эффективно для отгрузки. Этот стимул вредит выполнению аннулируемого контракта, когда внешний вариант поставщика непривлекателен (например, во время экономического спада), а затраты страховщика на мониторинг невелики (что может быть результатом превосходных информационных систем и/или аналитических возможностей). В этих обстоятельствах поставщик может предпочесть не подлежащие расторжению контракты, которые служат обязательством страховщика обеспечить покрытие. Это предпочтение не подлежащим отмене контрактам дает возможное объяснение большему энтузиазму в отношении не подлежащего отмене покрытия после Великой рецессии 2008 года. Мы также обнаружили, что контракты, сочетающие в себе преимущества отменяемых и не отменяемых функций, могут еще больше повысить ценность СТК.

Наконец, по сравнению со случаем, когда информация поддается проверке, когда информация, собранная страховщиком, не поддается проверке и подвержена стратегическим манипуляциям, аннулируемые контракты в целом становятся более привлекательными по сравнению с не подлежащими отмене контрактами. Это связано с тем, что действия страховщика по финансированию служат достоверным сигналом (непроверяемой) информации. Результат также предполагает, что контракты, не подлежащие аннулированию, станут относительно более привлекательными, поскольку полученная информация может быть достоверно передана по другим каналам (например, интеграция  в ИТ).

Вклад этой статьи двоякий. Во-первых, мы моделируем и количественно оцениваем ценность СТК, а также выделяем его оперативную роль, связанную с решениями по мониторингу и отгрузке. Анализ этой величины позволяет нам предложить теоретическое обоснование распространенности различных контрактов на ИТК, а также охарактеризовать их относительные достоинства. Во-вторых, полученные результаты проливают свет на то, как страховщики могут лучше разрабатывать и внедрять наиболее подходящие типы контрактов для клиентов. Наш анализ также предоставляет специалистам по поставкам информацию о том, как выбрать наиболее подходящий страховой полис торгового кредита.

\section{Сопутствующая литература}

В прикладном плане наша работа связана с двумя основными направлениями литературы: страхование на стыке с управлением операциями, финансами и управлением рисками.

В литературе по страхованию основное внимание уделяется последствиям асимметричной информации о застрахованной стороне, которая проявляется в форме неблагоприятного отбора (например, Ротшильд и Стиглиц 1976) или морального риска (например, Шавелл 1979). Наша статья связана с обоими подходами. Во-первых, в литературе по моральному риску в условиях страхования застрахованный и страховщик одинаково информированы во время заключения договора, однако застрахованная сторона может предпринимать скрытые действия после заключения договора, и оптимальный договор страхования предназначен для стимулирования более эффективных действий с использованием франшизы в качестве широко используемого механизма. Винтер (2013) дает отличный обзор литературы по страхованию от морального риска. Наша работа вносит следующий вклад. Во-первых, насколько нам известно, все существующие работы моделируют только моральный риск, связанный с действиями застрахованного лица. Однако в нашей статье, поскольку страховщик торгового кредита играет активную роль в мониторинге рисков, оптимальный контракт должен смягчать не только моральный риск поставщика (решение о доставке), но и моральный риск, связанный с усилиями страховщика по мониторингу. Таким образом, мы считаем, что традиционный контракт (не подлежащий отмене контракт с франшизой) может быть неэффективным для смягчения этих моральных рисков, и предоставление страховщику возможности отмены часто улучшает исполнение контракта.

Во-вторых, в зависимости от того, имеет ли место моральный риск до или после реализации фокальной неопределенности, в существующей литературе рассматривается либо только упреждающий моральный риск, такой как недостаточное инвестирование в меры предосторожности (H olmstrom 1979), либо только моральный риск ex-post, такой как перерасход средств на медицинское обслуживание, когда условия здоровья не могут быть согласованы (Zeckhauser 1970, Ma and Riordan 2002). Иными словами, наша модель учитывает как упреждающий моральный риск (усилия страховщика по мониторингу), так и постфактум риск (решение поставщика о доставке и решение страховщика об отмене по аннулируемым контрактам). Кроме того, усилия страховщика приводят к раскрытию информации, ценность которой напрямую зависит от решения поставщика о доставке. Таким образом, эти две моральные опасности должны быть смягчены совместно, и результирующая динамика качественно отличается от тех, которые изучаются в литературе, и оправдывает различные формы контрактов в СТК.

В-третьих, в отличие от литературы, в которой существование морального риска не зависит от используемого контракта, мы изучаем ситуацию, в которой моральный риск является эндогенным для формы контракта. В частности, аннулируемый договор приводит к новому моральному риску, который позволяет страховщику отменить страховое покрытие, когда оно эффективно для отгрузки. Этот моральный риск, который отсутствует в не подлежащем отмене контракте, помогает объяснить, почему и когда не подлежащий отмене контракт может быть предпочтительной формой контракта.

Рассматривая случай, когда информация, которую получает страховщик, не поддается проверке (§7), наша статья также связана с проблемой неблагоприятного отбора в страховании. В литературе проблема неблагоприятного отбора возникает из-за личной информации застрахованного лица до заключения договора, и основной вопрос заключается в том, как страховщик может использовать контракты для проверки такой информации. Иначе говоря, наша модель показывает, что страховщик получает личную информацию после заключения договора. Таким образом, страховщик должен использовать достоверное сообщение для передачи своей личной информации, которая моделируется как сигнальная игра, встроенная в вышеупомянутую ситуацию морального риска. Наш результат здесь также проливает свет на выбор между различными формами контрактов.Подчеркивая операционную ценность СТК, наша работа также связана с быстрорастущей областью взаимодействия управления операциями, финансами и управлением рисками (например, Бабич и Собель 2004, Дада и Ху 2008, Боябатли и Токтай 2011). В рамках этого потока наша работа больше всего связана с документами о взаимодействии между страхованием и операциями. Донг и Томлин (2012) характеризуют оптимальную политику инвентаризации фирмы при наличии страхования от прерывания бизнеса (BI). Серпа и Кришнан (2017) также изучают BI-страхование, уделяя особое внимание его стратегической роли в смягчении последствий э"ффекта безбилетника". Подобно Серпе и Кришнану (2017), наша статья также построена на теоретико-игровой модели. Иными словами, мы изучаем страхование торговых кредитов, еще один сектор страхования, тесно связанный с управлением цепочками поставок. В связи с этим мы фокусируемся на стратегическом взаимодействии между поставщиком и страховщиком, который выполняет двойную роль мониторинга и сглаживания денежных средств, и изучаем его последствия для форм контрактов. Наша статья также связана с недавними работами о взаимодействии между конфликтом агентств, оперативными решениями и финансовыми контрактами (Alan and Gaur 2018, Tang et al. 2018, Babich et al. 2017), особенно теми, в которых основное внимание уделяется смещению рисков как основному источнику финансовых трений. В нашей статье, как только договор страхования заключен, поведение двух договаривающихся сторон может отклоняться от первого наилучшего, аналогично поведению, связанному с изменением риска, изученному в вышеприведенных работах. Однако наша статья отличается от вышеперечисленных не только контекстом (СТК), но и лежащим в основе экономическим механизмом: во-первых, поскольку расторгаемые контракты предполагают обращение к обновленной информации, поведение, изменяющее риск, может возникать у обеих сторон в игре; во-вторых, в нашей статье поведение, изменяющее риск, возникает из обновленной информации, генерируемой эндогенно моральным риском страховщика. Таким образом, в нашей модели смещение риска взаимодействует с моральным риском. Кроме того, эта работа связана с работами по торговому кредиту (Babich and Tang 2012, Kouvelis and Zhao 2012, Peura et al. 2017, Yang and Birge 2018, Chod et al. 2019, Devalkar and Krishnan 2019), а также об операционных и финансовых средствах снижения риска цепочки поставок (Gaur and Seshadri 2005, Swinney and Netessine 2009, Yang et al. 2009, Turcic et al. 2015).


Концептуально, поскольку статья отражает моральный риск как со стороны страховщика, так и со стороны поставщика, наша статья связана с литературой по двойному моральному риску, которая была изучена в экономике (Бхаттачария и Лафонтен, 1995) и в операционной деятельности (Корбетт и др., 2005, Роэлс и др., 2010, Джайн и др., 2013). В целом, в этой литературе изучаются моральные риски, которые фактически являются одновременными, и, таким образом, основное внимание уделяется контрактам, основанным на производительности, и контрактам на совместное использование, которые являются статичными по своей природе и не требуют или не обладают способностью включать какие-либо обновления информации. Напротив, в нашей проблеме моральные опасности проявляются последовательно. В частности, усилия страховщика по мониторингу приводят к информационным выгодам, которые могут привести к последующим моральным рискам поставщика и страховщика; следовательно, мы рассматриваем форму договора с регрессом (возможность отмены), включив обновление, созданное с помощью мониторинга. Эта реактивная роль в точности соответствует цели, которой служит функция отмены в контрактах СТК.

\section{Модель}

Модель фокусируется на стратегическом взаимодействии между поставщиком (она), который предлагает торговый кредит своему покупателю, и страховщиком (он), который предлагает поставщику продукт СТК, который защищает ее в случае невыполнения покупателем платежа. Покупатель не принимает никаких решений в нашей модели.

\subsection{Поставщик и покупатель}

Поставщик получает заказ от покупателя, который соглашается приобрести у поставщика одну единицу товара по кредитной цене $r$, но склонен к неплатежу.
То есть покупатель обязан выплатить поставщику сумму $r$ в определенный момент времени после поставки товара; однако существует риск того, что покупатель может не выполнить платеж.
Пусть риск неисполнения обязательств покупателем при подписании контракта будет равен $\overline{\beta}$ . В случае неисполнения обязательств поставщик не получает денег от покупателя.

Чтобы защитить себя от неплатежей, продавец может приобрести страховку.

После подписания договора страхования, если поставщик узнает, что риск дефолта покупателя снизился,он может затем отказаться от поставки товара в кредит и вместо этого избавиться от товара по альтернативному каналу (внешний опцион поставщика) по цене $r_{0}$,
которая, как предполагается, ниже $(1 - \overline{\beta})r$. То есть покупатель априори кредитоспособен. Таким образом, предложение торгового кредита покупателю по
цене $r$ является более выгодным, чем внешний вариант, если поставщик не испытывает финансовых ограничений.

Цель поставщика состоит в том, чтобы максимизировать ожидаемую выплату, которая включает поступающую выручку и выплаты по страховым случаям, а также исходящую страховую премию и расходы, связанные с финансированием. Без потери общности безрисковая процентная ставка нормализуется до нуля.  Как документально подтверждено в финансовой литературе (Kaplan and Zingales 1997, Hennessy and Whited 2007, Shleifer and Vishny 2011), когда фирмы сталкиваются с дефицитом денежных средств, например, из-за дефолта покупателя, они несут расходы на внешнее финансирование из-за различных несовершенств финансового рынка, таких как транcакционные издержки (например, при продаже активов). Существование таких затрат требует, чтобы фирмы управляли неопределенностью денежных потоков с помощью различных инструментов управления рисками, таких как хеджирование и страхование (Froot et al. 1993, Dong and Tomlin 2012). В частности, мы предполагаем, что стоимость финансирования поставщика равна $L(x) = l(T - x)^{+}$, где $x$ представляет чистый денежный поток поставщика (конец периода), равный его доходу за вычетом страховой премии и страховой франшизы (если применимо), а $l$ -- предельные затраты на финансирование, которые несет поставщик, если $x$ не достигает определенного порога $T$, который отражает серьезность финансовых ограничений поставщика. Эта модель затрат на финансирование представляет собой абстракцию различных проблем, с которыми фирмы сталкиваются в реальности. Несмотря на финансовые ограничения в конце периода $T$, фирма обладает достаточной краткосрочной ликвидностью для покрытия страховой премии в середине периода.

Со стороны покупателя мы предполагаем, что кредитный риск покупателя развивается в период между моментом
заключения поставщиком и покупателем договора купли-продажи в кредит и моментом отгрузки поставщиком
заказа. В частности, к тому времени, когда поставщику необходимо решить, отправлять заказ или нет,
риск дефолта покупателя может быть на одном из трех уровней: низкий, средний или высокий, с соответствующей
вероятностью дефолта  $ \beta_{1}^{'}, \beta_{2}^{'}$ и $\beta_{3}^{'}$, где $0 \leq \beta_{1}^{'} \leq \beta_{2}^{'} \leq \beta_{3}^{'}$. Вероятность того, что вероятность дефолта покупателя равна $\beta_{i}^{'}$ равна $\theta_{i}^{'}$, где $\sum_{i}\theta_{i}^{'} = 1$ и $\overline{\beta} = \sum_{i}\theta_{i}^{'}\beta_{i}^{'}$. СТК играет важную
роль в получении информации (“сигналов”) относительно этого развивающегося риска дефолта.

\subsection{Страховщик торгового кредита и мониторинг рисков}

Чтобы охватить рыночную структуру отрасли СТК и сосредоточиться на операционных последствиях , мы предполагаем, что страховщик нейтрален к риску, не сталкивается с ограничениями ликвидности и работает на конкурентном страховом рынке. Поэтому страховщик готов предлагать договоры страхования до тех пор, пока премия покрывает его ожидаемую стоимость.

Отличительной особенностью нашей модели является то, что мы фиксируем действия страховщика по мониторингу рисков. В частности, мы предполагаем, что после заключения договора страхования страховщик принимает решение о том, следует ли прилагать непроверяемые усилия по мониторингу при стоимости $c \geq 0$. Приложив усилия с вероятностью $\lambda \in (0,1)$, страховщик может своевременно получить обновленную информацию (“сигналы”) о развивающемся риске дефолта покупателя до принятия поставщиком решения о доставке. В частности, сигналы, которые получает страховщик, могут быть классифицированы на три категории, соответствующие трем уровням риска дефолта покупателя (низкий, средний и высокий), как описано выше: сигнал в группе низкого риска $(i = 1)$ отражает, что покупатель действует как обычно с риском дефолта под контролем, включая случай, когда не обнаружено событий, усугубляющих риск. Сигнал, воплощающий средний риск $(i = 2)$, отражает сценарий, в котором есть некоторые тревожные признаки по отношению к кредитоспособности покупателя. Наконец, сигнал в группе высокого риска $(i = 3)$ показывает убедительные доказательства того, что продажа кредита является чрезмерно рискованной (например, подача заявления о банкротстве). Однако с вероятностью $(1 - \lambda)$ страховщику не удается получить какой-либо сигнал, который нельзя отличить от сценария, когда не наблюдается отрицательной информации $(i = 1)$. Таким образом, исходя из правила Байеса, при получении сигнала, принадлежащего группе $i = 1$, или вообще никакого сигнала, которые не отличаются друг от друга, вероятность дефолта покупателя по умолчанию составляет $\beta_{1} = (1 - \lambda)\overline{\beta} + \lambda\beta_{1}^{'}$ . Вероятность того, что этот сценарий произойдет, равна $\theta_{1} = (1 - \lambda) + \lambda\theta_{1}^{'}$ . При получении сигнала среднего или высокого уровня апостериорная вероятность дефолта покупателя $\beta_{i} = \beta_{i}^{'}$ для $i = 2, 3$. Вероятность того, что эти сценарии произойдут, равна $\theta_{i} = \lambda\theta_{i}^{'}$ для $i = 2, 3$ соответственно. С другой стороны, если страховщик решает не прилагать усилий, никакой обновленной информации не наблюдается, и вероятность дефолта последующего покупателя остается такой же, как и предыдущая, $\overline{\beta}$ . Мы отмечаем, что приведенная выше модель генерации информации аналогична модели, описанной в Stein (2002), и согласуется с классической литературой о моральном риске в том, что нельзя использовать результат для определения того, приложил ли страховщик усилия, и, следовательно, усилия не могут быть напрямую связаны.

Наконец, в зависимости от того, поддается ли полученная информация проверке или нет, мы рассматриваем два сценария относительно того, как страховщик делится информацией с поставщиком. В §5-6 мы фокусируемся на случае, когда информация поддается проверке и не может быть стратегически изменена страховщиком. Это в значительной степени согласуется с нашим пониманием практики (например, ИТ-системы страховщика и поставщика частично связаны), и это позволяет нам сосредоточиться на моральном риске, связанном с выполнением страховщиком своей контрольной роли. В §7 мы рассматриваем сценарий, в котором полученная информация не поддается проверке, и страховщик может стратегически манипулировать информацией.

\subsection{Договоры СТК}

Руководствуясь отраслевой практикой (Ассоциация британских страховщиков 2016, AIG 2013, Jones 2010, Thomas 2013), мы в основном фокусируемся на следующих двух типах контрактов СТК с соответствующей последовательностью событий, как показано на рисунке:
\\
\\

• Не подлежащий отмене СТК: В соответствии с не подлежащим отмене контрактом поставщик предлагает франшизу $\delta$ и премию $p$, и страховщик принимает решение принять или отклонить контракт. После подписания контракта поставщик выплачивает страховщику p, и страховщик может принять решение о проведении мониторинга по цене $c > 0$ и поделиться своими выводами с поставщиком. Однако, независимо от решения страховщика об усилиях и сигнала, который он получает, до тех пор, пока поставщик отправляет заказ, страховщик должен оплатить требование поставщика $r - \delta$ в случае невыполнения покупателем обязательств и заплатить ноль в противном случае.

• Аннулируемые СТК: Аннулируемые контракты отличаются от не аннулируемых в двух отношениях. Во-первых, контракт включает в себя не только $p$ и $\delta$, но и возврат премии $f \in \left[0, p \right]$. Во-вторых, страховщик имеет возможность отменить страховку в любое время до отгрузки товара. Если он отменяет покрытие, страховщик возмещает поставщику $f$ и устраняет его подверженность кредитному риску покупателя. Если он не отменяет, страховщик выплачивает поставщику $r - \delta$, если покупатель не выполняет свои обязательства. Мы отсылаем читателей к TATA-AIG (2017) для ознакомления с политиками, используемых в типичном аннулируемом контракте.

Поскольку предполагается, что рынок страхования является конкурентным, равновесный договор страхования-это тот, который приводит к наибольшей выплате поставщику, который мы определяем как оптимальный контракт. В связи с этим мы говорим, что решение о доставке является эффективным тогда и только тогда, когда поставщик доставляет по всем сигналам $i$ с $(1 - \beta_{i})r \geq r_{0}$, политика доставки в соответствии с централизованным эталоном, как показано ниже.

Наконец, сосредоточиться на связи между операционными и финансовыми аспектами моделим ы делаем два технических предположения. Первый из них:
\\

 Предположение 1. $r_{0} > (1 - \beta_{3})r$.
\\

Это предположение утверждает, что внешний опцион более выгоден, чем доставка покупателю, склонному к риску, поскольку в противном случае внешний опцион не имеет ценности и получение обновленной информации не имеет никаких операционных последствий. Кроме того, мы отмечаем , что, поскольку $\beta_{1} < \overline{\beta}$, более раннее условие, что $r_{0} > (1 - \overline{\beta})r$ также подразумевает, что при получении обновленной информации всегда эффективно отправлять по сигналу $i = 1$. В совокупности эти предположения гарантируют, что получение обновленной информации имеет эксплуатационную ценность. Кроме того, отметим, что эффективное решение при $i = 2$ зависит от привлекательности $r_{0}$ относительно $\beta_{2}$. В остальной части статьи мы относим случай $r_{0} > (1 - \beta_{2})r$ к привлекательному внешнему варианту или, что эквивалентно, неэффективно отправлять при $i = 2$, а случай $r_{0} \leq (1 - \beta_{2})r$ к непривлекательному внешнему варианту, т. Е. Доставка при $i = 2$ неэффективна.


Во-вторых, мы предполагаем, что поставщик несет финансовые расходы только в том случае, если он отправляет товары покупателю, который впоследствии не выполняет свои обязательства. Достаточное условие для обеспечения этого характеризуется следующим образом.
\\

Предположение 2. $T \in [0, r_{0} - (r - r_{0}))$.
\\

Поскольку страховая премия никогда не превышает $(r - r_{0})$, предположение 2 гарантирует, что поставщик
не понесет финансовых затрат при покупке страховки и последующей продаже стороннему опциону.

\section{Двойная роль СТК и его потенциальная ценность}
Прежде чем анализировать эффективность различных контрактов на СТК, мы сначала устанавливаем потенциальную экономическую ценность СТК путем сравнения двух контрольных показателей: централизованного контрольного показателя, когда действия поставщика и страховщика контролируются центральным лицом, принимающим решения, которое финансово не обременено (как страховщик) и максимизирует сумму выплат обеих сторон; и контрольного показателя без страхования, когда поставщик не имеет доступа к какому - либо продукту СТК. В наших условиях централизованный результат эквивалентен первому наилучшему результату, т. Е. Когда игроки максимизируют собственные интересы, но все действия ограничиваются (без морального риска). Разница в профиците системы между двумя базовыми сценариями представляет потенциальное значение СТК, которое формализуется в результате ниже.
\\

Предположение 1. Потенциальная ценность СТК составляет $min\{ \overline{\beta}L(0), (1 -\overline{\beta})r - r_{0} \} + (\phi - c)^{+} $, где $\phi = \sum_{i = 1}^{3}\theta_{i}\left[ r_{0} - (1-\beta_{i})r \right]^{+}$ -- значение опции консолидации отгрузки.
\\

Предложение 1 показывает, что потенциальное значение СТК имеет две составляющие: значение сглаживания денежного потока

$min\left[ \overline{\beta}L(0),
(1 -\overline{\beta})r - r_{0} \right]$ и значение мониторинга $(\phi - c)^{+}$. Во-первых, как и другие формы страхования, СТК сглаживает денежный поток страховой компании и снижает затраты, связанные с финансовыми ограничениями поставщика. В частности, без страхования из - за ограничения финансирования возникают два возможных исхода: поставщик либо оказывает услуги рискованному покупателю при любых обстоятельствах и выплачивает (ожидаемую) стоимость финансирования $\overline{\beta}L(0)$, либо поставщик всегда отправляет свой внешний опцион и несет альтернативные издержки $(1 -\overline{\beta})r - r_{0}$. СТК устраняет такие затраты, сглаживая денежный поток поставщика. Интуитивно такое значение больше, когда поставщик более ограничен в финансовом отношении (больше $l$ или $T$ , и, следовательно, больше $L(0)$) или сталкивается с менее привлекательным внешним вариантом (ниже $r_{0}$).
Во-вторых, и это более важно для наших целей, СТК также играет роль мониторинга, которая уникальна в условиях СТК из-за превосходящих возможностей страховщика (по сравнению с поставщиком) в получении и анализе информации. В частности, прилагая дорогостоящие усилия, страховщик получает обновленную информацию о риске дефолта покупателя. Это позволяет поставщику отменить доставку после получения сигнала i таким образом, что доставка внешнему опциону генерирует более высокий выигрыш, чем первоначальному рискованному покупателю, т. е. $r_{0} > (1-\beta_{i})r$. Эта опция отмены доставки создает рабочее значение $\phi = \sum_{i = 1}^{3}\theta_{i}\left[ r_{0} - (1-\beta_{i})r \right]^{+}$. Всякий раз, когда это значение превышает затраты на мониторинг c, усилия по мониторингу эффективны. Чтобы избежать неинтересного случая, когда страховщику нецелесообразно прилагать усилия по мониторингу даже в отсутствие агентских проблем $(c > \phi)$, остальная часть статьи сосредоточена на регионах, где $(c \leq \phi)$.
В то время как потенциальная ценность обеих ролей СТК полностью реализуется в централизованной среде, в децентрализованной среде, где и страховщик, и поставщик действуют соответственно, чтобы максимизировать свои собственные интересы, ожидается, что реализованная ценность СТК будет зависеть от того, сможет ли договор страхования успешно выполнять две роли СТК. В частности, обратите внимание, что выполнение мониторинговой стоимости СТК зависит от стимула страховщика инвестировать в мониторинг, а также от готовности поставщика эффективно осуществлять отгрузку на основе обновленной информации. Эти два вопроса стимулирования взаимосвязаны, и вместе они выступают в качестве основной движущей силы эффективности различных контрактов на ОТК, что является предметом рассмотрения в следующих разделах. Там мы говорим, что поставщик получает полную стоимость СТК, если выплата поставщика по контракту СТК равна выплате по централизованному бенчмарку. В противном случае поставщик получает только частичную стоимость СТК.
\section{Аннулируемые контракты}
В этом разделе мы изучаем аннулируемые контракты и выясняем, могут ли они смягчить некоторые ограничения не подлежащих аннулированию контрактов. Очевидно, что возможность страховщика отменить покрытие имеет ценность только тогда, когда он получает обновленную информацию и когда он фактически использует эту возможность при определенных обстоятельствах. Таким образом, в этом разделе мы ограничиваемся теми договорами, по которым страховщик прилагает усилия по мониторингу и фактически отменяет покрытие при получении определенных сигналов.
\subsection{Взаимосвязь между отгрузкой поставщика и решением страховщика об отмене}
Чтобы определить оптимальный контракт, подлежащий отмене, мы сначала характеризуем политику доставки поставщика, а затем политику отмены страховщика по данному договору страхования $(p, \delta, f)$, где $f \in [0,p]$ - возмещение, которое страховщик выплачивает поставщику при отмене страхового полиса.
В зависимости от того, будет ли отменено ее покрытие, политика доставки поставщика может обсуждаться в двух сценариях. Во-первых, когда страховое покрытие не отменяется, выплата поставщика по сигналу $i$ составляет $r-p-\beta_{i}[\delta +L(r - \delta - p)]$, если он отгружает, и $r_{0}$, если он не отгружает. Следовательно, поставщик отправляет заказ тогда и только тогда, когда $\beta_{i} \leq \mathbb{B}(p, \delta)$, где $\mathbb{B}(\it{p}, \delta)$ - пороговый риск банкротства при котором поставщику безразлично, отгружать товар или нет по контракту $(p, \delta)$.
В случае отмены страховки по сигналу $i$ выплата поставщика составляет $r-p+f-\beta_[i][r+L(f-p)]$, если она отгружает, и $r_{0}-p+f$, если она этого не делает. Сравнивая две выплаты, поставщик отгружает товары тогда и только тогда, когда:
\begin{equation}
	\beta_{i} \leq \mathbb{B}_{c}(p,f):= \frac{r-r_{0}}{r+L(f - p)},
\end{equation}
где индекс $C$ означает, что покрытие отменено. Обратите внимание, что при $L() \geq 0$, $\mathbb{B}_{c} \leq (r - r_{0})/r$; это отражает предположение о том, как отмена покрытия может удержать поставщика от чрезмерной отгрузки.
В ожидании решения поставщика о доставке, описанного выше, страховщик принимает решение об отмене страхового покрытия. Поскольку страховщик должен вернуть $f$ поставщику при отмене, он никогда не отменяет, когда $\beta_{i} > \mathbb{B}$, для которого поставщик никогда не отправит заказ даже под страховым покрытием. Однако, если поставщик отгружает заказ, когда покрытие не отменено $\beta_{i} \leq  \mathbb{B}$, ожидаемая стоимость претензий для страховщика составляет $\beta_{i}(r - \delta)$. Балансируя эту стоимость и возмещение $f$, страховщик отменяет покрытие тогда и только тогда , когда $\beta_{i} \in [\mathbb{B}_{p}, \mathbb{B}]$, где
\begin{equation}
	\mathbb{B}_{p}(\delta,f):= \frac{f}{r-\delta}
\end{equation}
а индекс $P$ представляет собой политику отмены страховщиком. Таким образом, страховщик отменяет покрытие только в том случае, если риск дефолта покупателя находится в среднем диапазоне. На низком уровне риск невелик по отношению к возврату, в то время как на высоком уровне поставщик сам прекращает доставку. Тем не менее, верхний порог $\mathbb{B}$ может быть выше $\beta_{3}$, когда франшиза достаточно мала, и в этот момент политика отмены страховщиком вырождается в простую пороговую политику.
Лемма 3. По любому аннулируемому контракту, в котором страховщик прилагает усилия и фактически отменяет покрытие при определенных сигналах , $\mathbb{B} > max(\mathbb{B}_{C}, \mathbb{B}_{P})$, и поставщик отгружает товары тогда и только тогда, когда $\beta_{i} < max(\mathbb{B}_{C}, \mathbb{B}_{P})$.
Вышеприведенный результат предполагает, что политика доставки поставщика может быть полностью согласована с политикой отмены страховщиком (когда $\mathbb{B}_{C} \leq \mathbb{B}_{P}$), но она также может выполняться, когда покрытие отменено ($\mathbb{B}_{C} > \mathbb{B}_{P}$). В любом случае $\mathbb{B} > max(\mathbb{B}_{C}, \mathbb{B}_{P})$ подтверждает интуицию о том , что аннулируемый контракт побуждает поставщика отгружать кредитному покупателю более консервативно, чем без возможности аннулирования. Поскольку аннулируемый контракт, по которому страховщик никогда не отменяет покрытие, эквивалентен не подлежащему аннулированию контракту, мы сосредоточимся на аннулируемых контрактах, которые удовлетворяют лемме 3.
\subsection{Сила аннулируемого страхования}
Объединяя решения страховщика об отмене и решения поставщика о доставке, мы можем охарактеризовать условия, при которых аннулируемый контракт побуждает страховщика прилагать усилия по мониторингу (см. Лемму B. 1 в Приложении). Эти условия (например, (12) – (14)) имеют ту же структуру, что и в случае, не подлежащем отмене (лемма 1), но отличаются тем, что они включают политику отмены страховщика. Среди контрактов $(p,\delta,f)$, удовлетворяющих этим условиям, в интересах поставщика выбрать тот, который максимизирует его собственную отдачу, т. е.,
\begin{equation}
	max_{p,\delta \in [0,r], f \leq p} - p\sum_{i}\theta_{i}[r - \beta_{i}(\delta+L(r-p-\delta))]1_{\beta_{i}<\mathbb{B}_{P}}+\sum_{i}\theta_{i}[(1 - \beta_{i})r - \beta_{i}L(f-p)]1_{\beta_{i} \in [\mathbb{B}_{P}, \mathbb{B}_{C}]}
\end{equation}
\[
	+ r_{0}\sum_{i}\theta_{i}1_{\beta_{i}>max(\mathbb{B}_{P}, \mathbb{B}_{C})}
	 + f\sum_{i}\theta_{i}1_{\beta_{i} \in [\mathbb{B}_{P}, \mathbb{B})}
\]
Экв. (8) показывает, что при отмене контрактов, поставщика ожидается выплата состоит из пяти частей: страховая премия, ее чистый доход при поставке под покрытие $(\beta_{i} < \mathbb{B}_{P})$; что при транспортировке без покрытия $(\beta_{i} \in [\mathbb{B}_{P}, \mathbb{B}_{C}))$; ее пределами вариант, когда она не корабль $(\beta_{i} > max(\mathbb{B}_{P}, \mathbb{B}_{C}))$; И, наконец, возврат средств, когда покрытие отменяется $(\beta_{i} \in [\mathbb{B}_{P}, \mathbb{B}))$.
Решая эту программу оптимизации, мы характеризуем оптимальный контракт и соответствующие действия сторон на основе того, является ли внешний вариант поставщика привлекательным ($r_{0} \geq (1-\beta_{2})r$, предложение 3) и непривлекательным $r_{0} \leq (1-\beta_{2})r$, предложение 4 в §6.3).
\\
Предположение 3. При $r_{0} \geq (1-\beta_{2})r$ оптимальным является следующий аннулируемый контракт:
\\
\begin{equation}
	\delta=0; p=f=\frac{c}{\theta_{1}}+\beta_{1}r
\end{equation}
В соответствии с настоящим контрактом поставщик получает полную стоимость СТК только при отгрузке по цене $i = 1$. Соответственно, страховщик отменяет покрытие при получении сигнала $i = 2, 3$.
\\
По сравнению с выполнением оптимального контракта, не подлежащего отмене, как показано в предложении 2, мы замечаем, что при одной и той же области параметров ($r_{0} \geq (1-\beta_{2})r$) оптимальный контракт, не подлежащий отмене, всегда позволяет поставщику в полной мере пользоваться преимуществами СТК. Причина в следующем. Напомним, что неэффективность контрактов, не подлежащих аннулированию, в основном связана с двумя наборами конфликтов, которые тянут франшизу в разных направлениях: Конфликт внутри роли мониторинга и конфликт между ролью сглаживания и ролью мониторинга. С другой стороны, аннулируемый контракт имеет свои преимущества в разрешении обоих конфликтов. Чтобы увидеть, как аннулируемые контракты могут полностью восстановить роль мониторинга, вспомним из леммы 3, что поставщик более агрессивно отгружает товары под покрытием, чем без него. Таким образом, отменяя ее покрытие, страховщик может эффективно подтолкнуть поставщика к принятию более консервативной политики доставки. Иными словами, аннулируемый контракт (частично) передает контроль над решением о доставке от поставщика страховщику. Кроме того, поскольку страховщик не получает прямой выгоды от потенциала роста торговли, страховщик действительно склонен вести себя более консервативно, что согласуется с эффективной политикой доставки, когда поставщик сталкивается с привлекательным внешним вариантом. Следовательно, предоставление большего контроля страховщику эффективно сдерживает чрезмерную отгрузку.
Во-вторых, позволяя страховщику отменить покрытие, аннулируемый контракт частично разделяет две роли СТК, эффективно создавая два уровня франшизы: когда поставщик считает доставку эффективной $(i = 1)$, поставщик покрывается страхованием. При таких обстоятельствах сглаживающая роль СТК выполняется, поскольку номинальная франшиза $\delta$ остается низкой. Этот результат также согласуется с тем фактом, что на практике большинство аннулируемых контрактов включают нулевую франшизу, в то время как не подлежащие аннулированию контракты, как правило, имеют большую франшизу (Euler Hermes 2018). С другой стороны, когда доставка неэффективна $(i \geq 2)$, полная стоимость мониторинга СТК реализуется через отмену покрытия, что приводит к эффективной франшизе, равной $r$.
Таким образом, ценность аннулируемых контрактов заключается в их последовательном характере. В частности, право страховщика на отмену осуществляется после ознакомления с обновленной информацией. Такое действие ex post зависит от реализации сигнала и, таким образом, повышает значение мониторинга СТК. Кроме того, обратите внимание, что опция отмены может быть реализована только до принятия поставщиком решения о доставке, поскольку последующая отмена не соответствует какому-либо регрессу, создающему экономическую ценность. Это также согласуется с практикой, согласно которой страховщик не может отменить покрытие после отправки заказа.
\subsection{Опасность аннулируемых контрактов: чрезмерное аннулирование}
Хотя предложение 3 раскрывает некоторые достоинства аннулируемых контрактов, такие контракты не являются неограниченными. Как показано в следующем результате, полное восстановление стоимости СТК не всегда возможно при расторжении контрактов, когда внешний вариант поставщика непривлекателен.
Предположение 4. Когда внешний вариант поставщика непривлекателен $r_{0} \leq (1-\beta_{2})r$,
\\
1. Если $\beta_{2} \leq \overline{\beta}$ , существуют пороговые функции
$\phi_{1}^{C}(T) \leq \theta_{1}(\beta_{2} - \beta_{1})max(r - \frac{r-r_{0}}{\beta_{3}}, \frac{T}{1-\beta_{2}})$ таким образом, что выполнение оптимального аннулируемого контракта суммируется в следующей таблице:
(таблица)
2. Если $\beta_{2} > \overline{\beta}$ , существуют пороговые функции $\phi_{2}^{C}(T) \leq \phi_{3}^{C}(T)$, такие, что выполнение
оптимального аннулируемого контракта суммируется в следующей таблице:
(таблица)
Предложение 4 проиллюстрировано на рис. 3. Первым заметным результатом является то, что поставщик не может возместить полную стоимость СТК, когда затраты страховщика на мониторинг достаточно низки или когда средний сигнал $(\beta_{2})$ ухудшается по сравнению с предыдущим предположением.
Интересно, что причина, по которой низкая стоимость мониторинга вредит выполнению аннулируемого контракта, также проистекает из варианта отмены страховщиком. Хотя такой вариант повышает гибкость контракта, гибкость предоставляется страховщику и, следовательно, не всегда может принести пользу поставщику. Действительно, такая гибкость создает дополнительный моральный риск со стороны страховщика. Подобно тенденции поставщика к чрезмерной отгрузке, как только контракт заключен, страховщик использует свой вариант отмены в соответствии со своими собственными наилучшими интересами, которые могут не полностью совпадать с интересами поставщика. В частности, как было показано ранее, для того, чтобы страховщик не отменил по сигналу $i$, его стоимость отмены (возврата $f$) превышает ожидаемые расходы по претензиям $\beta_{i}(r - \delta)$, т. е.
\begin{equation}
	f \geq \beta_{i}(r - \delta)
\end{equation}
Это ограничение относится к двум сценариям, в которых страховщик имеет тенденцию к чрезмерной отмене, т. Е. отмене при сигнале $i$, когда $(1-\beta_{i})r > r_{0}$. Во-первых, когда внешний вариант поставщика непривлекателен (маленький $r_{0}$), остается эффективным отгрузить в некотором $i$, когда $\beta_{i}$ относительно велик. Таким образом, более трудно удовлетворить (10) для такого $i$. Во-вторых, когда стоимость мониторинга страховщика низка, страховая премия $p$ также имеет тенденцию быть низкой. Поскольку на практике возврат $f$ ограничен $p$, (10) становится более строгим по мере уменьшения $c$, что также снижает ожидаемые затраты страховщика, сохраняя все остальное постоянным. При выполнении обоих условий, чтобы удовлетворить (10) для любого сигнала $i$ с $(1-\beta_{i})r \leq r_{0}$, оптимальные контракты демонстрируют одну из двух возможных особенностей: они либо имеют более высокую франшизу, тем самым увеличивая затраты на финансирование; либо они имеют более высокую премию, чтобы обеспечить более высокий возврат, что оставляет арендную плату страховщику. Эти контракты отражены в регионе RI/FC на рисунке 3. По мере дальнейшего снижения c становится слишком дорогостоящим стимулировать страховщика к принятию эффективной политики отмены. Вместо этого он отменяет покрытие при $i = 2, 3$, хотя доставка при $i = 2$ эффективна. В ответ поставщик выбирает между двумя альтернативами в зависимости от своих финансовых ограничений. Когда поставщик меньше беспокоится о стоимости финансирования (малый T), он следует эффективной политике доставки, что означает, что он отгружает по цене $i=2$ незастрахованных (регион UI). Однако для крупных Т поставщик не может позволить себе доставку без страховки. Таким образом, она не отправляется при $i = 2$, даже если она эффективна (регион США). С такой дилеммой часто сталкиваются поставщики. Например, в 2009 году Total Security Systems, небольшой поставщик британских охранных компаний, должен был решить, отправлять ли заказ на сумму 100 000 фунтов стерлингов после того, как их СТК был отменен. Несмотря на то, что они считали, что риск дефолта покупателя невелик, они отменили заказ, поскольку не хотели отправлять незастрахованные товары (Stacey 2009).
Наконец, мы отмечаем, что неэффективность расторгаемых контрактов более выражена, когда кредитный риск по среднему сигналу высок $(\beta_{2} \geq \overline{\beta})$. В этом случае страховщик имеет большую тенденцию к отмене при $i = 2$, и поставщик вынужден сдавать больше арендной платы страховщику, чтобы предотвратить чрезмерную отмену. Таким образом, стороны с большей вероятностью примут либо неэффективную политику отмены, либо неэффективную политику доставки.
\subsection{Выбор контрактов и улучшения}
Непосредственно сравнивая оптимальные не подлежащие отмене и аннулируемые контракты, мы отмечаем, что, когда внешний вариант поставщика привлекателен, т. е. $r_{0} \geq (1-\beta_{2})r$, поставщик всегда (слабо) предпочитает аннулируемый контракт, который восстанавливает полную стоимость СТК, по сравнению с не подлежащим отмене контрактом. Выполнение аннулируемого контракта строго доминирует, когда стоимость мониторинга высока, т. е. $c > \phi_{FV}^{N}(T)$ (в предложении 2). С другой стороны, предпочтение поставщика при наличии непривлекательного внешнего варианта сводится к следующему.
\\
Предположение 5. Когда внешний вариант непривлекателен $r_{0} < (1-\beta_{2})r$:
1. Для $\beta_{2} \leq \overline{\beta}$ существует пороговая функция $\phi_{L}(T)$, такая, что поставщик предпочитает не аннулируемый контракт с мониторингом, когда $c \leq \phi_{L}(T)$, и аннулируемый, когда $c > \phi_{L}(T)$.
2. Для $\beta_{2} > \overline{\beta}$, существует пороговых функций $\phi_{H,1}(T) \leq \phi_{H,2}(T)$ такое, что поставщик предпочитает не может быть расторгнут договор с мониторингом после того, как $c \leq \phi_{H,1}(T)$ и не может быть расторгнут договор
без контроля при $c \geq \phi_{H,2}(T)$.
Приведенное выше предложение подтверждает, что, когда внешний вариант непривлекателен, поставщик в целом предпочитает не подлежащий отмене контракт, когда стоимость мониторинга достаточно низка.6 Этот результат перекликается с недавним появлением не подлежащих отмене контрактов. Благодаря внедрению передовых информационных систем и другим технологическим достижениям затраты страховщика на мониторинг в настоящее время значительно ниже, чем в прошлом. Это потенциально приводит к снижению страховых премий, и, как уже упоминалось ранее, низкая премия обычно приводит к чрезмерной отмене страховщиком, что также согласуется с тем фактом, что страховщики СТК были обвинены в необоснованной отмене страхового покрытия. Кроме того, во время финансового кризиса поставщики столкнулись со сложными рыночными условиями и часто были лишены привлекательных внешних вариантов. Такие ситуации делают поставщиков особенно уязвимыми перед тенденцией страховщиков к чрезмерному аннулированию, что делает не подлежащие аннулированию контракты более желательным выбором.
Как было показано в предыдущих результатах, при выборе контрактов СТК поставщику необходимо измерить и сравнить ценность гибкости, обеспечиваемой отменой, с ценностью обязательства, воплощенного в не подлежащем отмене покрытии. Интуитивно понятно, что аннулируемый контракт с частью покрытия, не подлежащей аннулированию, который сочетает в себе эти два преимущества, может еще больше улучшить производительность СТК. Этот тип контракта соответствует формирующейся отраслевой практике добавления не подлежащего отмене покрытия к аннулируемому контракту, который иногда называют дополнительным покрытием по аннулируемому покрытию (Страховой журнал 2012). Такой частично аннулируемый контракт аналогичен аннулируемому контракту с одной лишь разницей: при аннулировании страховщик может отменить только часть страхового покрытия, в то время как оставшаяся часть не подлежит аннулированию. Моделируя эту инновационную форму контракта, мы обнаруживаем, что такие контракты еще больше расширяют регион, в котором поставщик получает полную стоимость СТК (подробнее см. Предложение B. 2 в Приложении). Наш анализ показывает, что эта инновация повышает ценность по двум каналам: во-первых, она защищает поставщика от финансовых затрат за счет не подлежащего отмене покрытия; во-вторых, она удерживает страховщика от чрезмерной отмены.
\section{Последствия непроверяемой информации}
Чтобы сосредоточиться на моральном риске, связанном со страховщиком, выполняющим свою роль мониторинга, мы предполагаем в предыдущих разделах, что, как только страховщик получает обновленную информацию с помощью мониторинга, он делится информацией с поставщиком без искажений. Это предположение в значительной степени согласуется с нашим пониманием практики (например, ИТ - системы страховщика и поставщика частично связаны, что затрудняет искажение информации). Однако существует вероятность того, что страховщик может стратегически манипулировать полученной информацией, и это, вероятно, более вероятно, когда информация получена из непубличных источников и, следовательно, трудно проверить. В этом разделе рассматриваются последствия такой непроверяемой информации. Модель, включение договоров СТК идентично тому, что указано в §3, за исключением того, что при мониторинге страховщик может исказить информацию, которую он собирает.
В этом случае мы сначала отметим, что потенциальная стоимость СТК такая же, как показано в предложении 1, поскольку информация непосредственно доступна страховщику. Что касается исполнения контрактов СТК, когда страховщик максимизирует свои личные интересы, интуитивно понятно, что у страховщика всегда есть стимул искажать информацию, чтобы удержать поставщика от доставки заказа, что избавило бы его от любой ответственности по претензиям. Предвидя это, поставщик считает бесполезной любую информацию, которую передает страховщик. Это, в свою очередь, не даёт страховщику прилагать усилия по мониторингу и, таким образом, ослабляет контрольную роль СТК. Таким образом, в соответствии с договором, не подлежащим расторжению, когда у страховщика нет других надежных средств для передачи собранной информации поставщику, у страховщика нет стимула прилагать усилия и, таким образом, он не выполняет роль мониторинга. В результате получается оптимальный контракт, не подлежащий отмене, как показано в лемме 2, и ясно, что непроверяемая информация вредит выполнению контрактов, не подлежащих отмене.
Однако влияние непроверяемой информации на аннулируемые контракты является более сложным. Во-первых, мы отмечаем, что аннулируемые контракты предоставляют страховщику инструмент, который потенциально может передать собранную информацию поставщику, а именно его действие по аннулированию. Интуиция подсказывает, что затраты, понесенные страховщиком на отмену, зависят от информации, которую он получает: когда информация положительна (низкий риск дефолта), стоимость отмены высока, а когда отрицательна (высокий риск дефолта), отмена менее затратна. Таким образом, когда поставщик замечает, что его страховое покрытие отменено, он, скорее всего, будет убежден, что информация, которую собирает страховщик, является отрицательной.
Чтобы формализовать вышеприведенный аргумент, мы моделируем взаимодействие между страховщиком и поставщиком после заключения контракта и страховщиком, приложившим усилия по мониторингу, как сигнальную игру. Страховщик (отправитель) передает свою личную информацию поставщику (получателю) с действием отмены в качестве сообщения. Мы отсылаем читателей к Райли (2001) для обзора сигнальных игр в экономической литературе, а также к Лаю и др. (2011), Бакши и др. (2015) и Тану и др. (2018) для применения в литературе по ОМ. Соответствующая концепция равновесия-это Идеальное байесовское равновесие (PBE). В равновесии, наблюдая за действием страховщика по отмене (сообщением), поставщик формирует свое заднее убеждение, основанное на информации, которую собирает страховщик, используя правило Байеса. После этого поставщик принимает соответствующее решение об отгрузке. Специфично для установки СТК, мы фокусируемся на равновесиях с полуотделением, поскольку количество возможных сообщений (отмена или не отмена) меньше, чем количество информационных классов $(i = 1, 2, 3)$, и, таким образом, невозможно полностью отделить каждый класс информации с помощью различных действий. Другой возможный тип равновесий-это равновесия объединения. Однако, поскольку у страховщика нет стимула осуществлять мониторинг в условиях такого равновесия, в результате аннулируемого контракта будет (слабо) доминировать оптимальный не подлежащий аннулированию контракт.
Изучая эту сигнальную игру (мы отсылаем читателей к лемме B. 2 в Приложении для получения подробной информации), мы обнаруживаем, что в соответствии с оптимальным аннулируемым контрактом страховщик принимает интуитивно понятную политику отмены порога: он отменяет контракт при получении отрицательной информации и не отменяет при получении положительной информации. В ответ поставщик отправляет товар, тогда и только тогда его покрытие не отменено. Для равновесного результата в сигнальной игре анализ, касающийся решения страховщика приложить усилия по мониторингу и выбора поставщиком условий контракта, аналогичен анализу, приведенному в §6. Мы представляем исполнение полученного оптимального контракта следующим образом.
Предложение 6. Когда информация, собранная страховщиком, не поддается проверке,
1. Для привлекательного внешнего варианта $r_{0} \geq (1-\beta_{2})r$ поставщик получает полную стоимость СТК по оптимальному аннулируемому контракту;
2. Для непривлекательного внешнего варианта $r_{0} < (1-\beta_{2})r$:
i) Если $\beta_{2} \leq \overline{\beta}$ , то существуют пороговые функции $\phi_{1}^{U}(T) \geq \phi_{FV}^{U}(T = \frac{\theta_{1}(\beta_{2} - \beta_{1})T}{1-\beta_{2}})$ такие, что выполнение оптимального контракта, подлежащего расторжению, суммируется в следующей таблице.
(таблица)
ii) Если $\beta_{2} > \overline{\beta}$ , существуют пороговые функции $\phi_{2}^{U}(T)\leq \phi_{3}^{U}(T)$, такие, что выполнение оптимального аннулируемого контракта суммируется в следующей таблице.
(таблица)
Сравнивая оптимальный аннулируемый контракт в соответствии с непроверяемой информацией с контрчастями в соответствии с проверяемой информацией (предложения 3 и 4), мы замечаем, что аннулируемый контракт остается эффективным при наличии непроверяемой информации, когда поставщик сталкивается с привлекательным внешним вариантом $r_{0} \geq (1-\beta_{2})r$. Однако при наличии непривлекательного внешнего варианта $r_{0} < (1-\beta_{2})r$, в то время как выполнение оптимального аннулируемого контракта следует структуре, аналогичной проверяемому информационному случаю, заметны два различия. Во - первых, оптимальный контракт при непроверяемой информации позволяет поставщику получить полную выгоду от СТК в более широком диапазоне затрат на мониторинг. Это связано с тем, что, когда обновленная информация не поддается проверке, поставщик вынужден реагировать на действия страховщика, а не на саму информацию. Таким образом, страховщик получает более прямой контроль и, таким образом, может частично смягчить проблему агентства поставщика. Во-вторых, непроверяемая информация также может нанести ущерб поставщику, что интересно, с той же силой. Конкретно, когда страховщик отменяет как при $i = 2$, так и при 3, поставщик не может различить эти два случая и, таким образом, теряет потенциально ценную возможность отгрузки не застрахованных (Регион UI в предложении 4).
Наконец, предложение 7 сравнивает вышеупомянутый аннулируемый контракт с не подлежащим аннулированию контрактом.
Предположение7. Когда информация, полученная страховщиком, не поддается проверке,
1. для $\beta_{2} \geq \overline{\beta}$ поставщик предпочитает расторгаемый контракт;
2. для $\beta_{2} < \overline{\beta}$ существует пороговая функция $\phi_{H}^{U}(T)$, такая, что поставщик предпочитает аннулируемый
контракт тогда и только тогда, когда $c \leq \phi_{H}^{U}(T)$.
По сравнению с результатами, полученными в рамках проверяемой информации (предложение 5), аннулируемые контракты, скорее всего, будут доминирующей формой заключения контрактов, когда собранная информация не поддается проверке, особенно когда затраты на мониторинг невелики. Это связано с тем, что в дополнение к своей роли в снижении морального риска вариант отмены, который присутствует только в аннулируемых контрактах, служит надежным сигналом для страховщика, чтобы сообщить частную информацию. Иными словами, если у страховщика есть другие каналы, по которым он может достоверно передавать собранную информацию поставщику (например, через ИТ-интеграцию или блокчейн), привлекательность контрактов, не подлежащих отмене, будет повышена, особенно при низких затратах на мониторинг. Это, возможно, дает еще одно объяснение растущей популярности контрактов, не подлежащих отмене.
\section{Заключение}
СТК является общепринятым инструментом управления рисками для поставщиков, которые предоставляют торговые кредиты своим покупателям. Несмотря на его широкое использование на практике, СТК в значительной степени игнорируется в академической литературе. В этой статье мы подчеркиваем операционную ценность СТК, связанную с ролью мониторинга, которую страховщик играет в настройке СТК. Мы также подчеркиваем моральные риски поставщика и страховщика, связанные с этой ролью. Сосредоточившись на отменяемости, отличительной черте контрактов СТК, мы определяем соответствующие преимущества и ограничения двух форм широко распространенных контрактов СТК: отменяемых и не отменяемых контрактов.
Наша статья может быть улучшена по различным направлениям. Например, сосредоточив внимание на взаимодействии страховщика и поставщика, мы предположили, что использование СТК не влияет на вероятность дефолта покупателя. Однако в некоторых случаях, если поставщик отзовет торговый кредит покупателя после аннулирования его страховки, покупатель может столкнуться с более серьезными ограничениями ликвидности и, следовательно, с большей вероятностью потерпит неудачу. Аналогичная динамика (без СТК) была недавно изучена в работах Бабича (2010) и Яна и др. (2015). Тем не менее, выбор контракта СТК при наличии эндогенного риска дефолта остается открытым вопросом. В связи с этим в документе условия торгового кредита между поставщиком и покупателем рассматриваются как экзогенно заданные. Однако наличие СТК может повлиять на условия между двумя торговыми сторонами, а также на другие варианты финансирования. Расширение текущей модели в этом измерении может стать перспективным направлением для будущих исследований. Наконец, хотя в статье основное внимание уделяется СТК, принципы, которые мы раскрыли, могут применяться в более общем плане. В частности, вариант отмены страховщиком может повысить стоимость других видов страхования, где страховщик лучше подготовлен для мониторинга страхового риска. Например, в качестве эмерджентного страхового продукта страхование условного прерывания бизнеса (CBI) возмещает упущенную выгоду и дополнительные расходы, возникшие в результате прерывания бизнеса на стороне клиента или поставщика, и, следовательно, также страхует риск, который не является внутренним для застрахованной стороны. Поэтому вполне возможно, что страховщик более эффективно контролирует и этот риск. Таким образом, наши результаты могут стать основой для управления рисками за пределами СТК.

\end{document}